%% Based on a TeXnicCenter-Template by Gyorgy SZEIDL.
%%%%%%%%%%%%%%%%%%%%%%%%%%%%%%%%%%%%%%%%%%%%%%%%%%%%%%%%%%%%%

%------------------------------------------------------------
%
\documentclass{amsart}
%
%----------------------------------------------------------
% This is a sample document for the AMS LaTeX Article Class
% Class options
%        -- Point size:  8pt, 9pt, 10pt (default), 11pt, 12pt
%        -- Paper size:  letterpaper(default), a4paper
%        -- Orientation: portrait(default), landscape
%        -- Print size:  oneside, twoside(default)
%        -- Quality:     final(default), draft
%        -- Title page:  notitlepage, titlepage(default)
%        -- Start chapter on left:
%                        openright(default), openany
%        -- Columns:     onecolumn(default), twocolumn
%        -- Omit extra math features:
%                        nomath
%        -- AMSfonts:    noamsfonts
%        -- PSAMSFonts  (fewer AMSfonts sizes):
%                        psamsfonts
%        -- Equation numbering:
%                        leqno(default), reqno (equation numbers are on the right side)
%        -- Equation centering:
%                        centertags(default), tbtags
%        -- Displayed equations (centered is the default):
%                        fleqn (equations start at the same distance from the right side)
%        -- Electronic journal:
%                        e-only
%------------------------------------------------------------
% For instance the command
%          \documentclass[a4paper,12pt,reqno]{amsart}
% ensures that the paper size is a4, fonts are typeset at the size 12p
% and the equation numbers are on the right side
%
\usepackage{amsmath}%
\usepackage{amsfonts}%
\usepackage{amssymb}%
\usepackage{mathrsfs}
%(wjd) added stmaryrd and enumerate packages
\usepackage{stmaryrd,enumerate}
\usepackage{graphicx}
\usepackage{comment}
%------------------------------------------------------------
% Theorem like environments
%
\theoremstyle{plain}
\newtheorem{theorem}{Theorem}
\theoremstyle{definition}
\newtheorem{acknowledgement}{Acknowledgement}
\newtheorem{algorithm}{Algorithm}
\newtheorem{axiom}{Axiom}
\newtheorem{case}{Case}
\newtheorem{claim}{Claim}
\newtheorem{conclusion}{Conclusion}
\newtheorem{condition}{Condition}
\newtheorem{conjecture}{Conjecture}
\newtheorem{corollary}{Corollary}
\newtheorem{criterion}{Criterion}
\newtheorem{definition}{Definition}
\newtheorem{example}{Example}
\newtheorem{exercise}{Exercise}
\newtheorem{lemma}{Lemma}
\newtheorem{fact}{Fact}
\newtheorem{notation}{Notation}
\newtheorem{problem}{Problem}
\newtheorem{proposition}{Proposition}
\newtheorem{remark}{Remark}
\newtheorem{solution}{Solution}
\newtheorem{summary}{Summary}
\newtheorem{tool}{Tool}
\theoremstyle{definition}
\newtheorem{Example}{Example}
\numberwithin{equation}{section}
%--------------------------------------------------------
%(wjd) A few useful macros 
\renewcommand{\phi}{\ensuremath{\varphi}}
\newcommand{\core}{\ensuremath{\operatorname{core}}}
\newcommand{\Sub}{\ensuremath{\operatorname{Sub}}}
\newcommand{\<}{\ensuremath{\langle}}
\renewcommand{\>}{\ensuremath{\rangle}}
\newcommand{\lb}{\ensuremath{\llbracket}}
\newcommand{\rb}{\ensuremath{\rrbracket}}
%(wjd) I think it's better to use \lb H, G \rb than [H,G], 
%      since single brackets typically denotes the commutator.
%      If you don't like this, uncomment the next two lines and
%      everything will change back to single brackets:
% \renewcommand{\lb}{\ensuremath{[}}
% \renewcommand{\rb}{\ensuremath{]}}

%
%(wjd) I prefer angled less or equal symbols, but you can
%      change these back by commenting the next 6 lines.
\renewcommand{\leq}{\ensuremath{\leqslant}}
\renewcommand{\nleq}{\ensuremath{\nleqslant}}
\renewcommand{\geq}{\ensuremath{\geqslant}}
\renewcommand{\lneq}{\ensuremath{\lneqslant}}
\renewcommand{\gneq}{\ensuremath{\gneqslant}}
\renewcommand{\ngeq}{\ensuremath{\ngeqslant}}
%
%(wjd) I prefer angled "normal or equal" symbol, but you can
%      change this back by commenting the next line.
\renewcommand{\unlhd}{\ensuremath{\trianglelefteqslant}}


\newcommand{\bA}{\ensuremath{\mathbf{A}}}
\newcommand{\sP}{\ensuremath{\mathscr{P}}}
%\newcommand{\ker}{\ensuremath{\operatorname{ker}}}
\newcommand{\Con}{\ensuremath{\operatorname{Con}}}
\newcommand{\Eq}{\ensuremath{\operatorname{Eq}}}
\newcommand{\rel}{\ensuremath{\mathrel}}
\newcommand{\meet}{\ensuremath{\wedge}}
\newcommand{\join}{\ensuremath{\vee}}
\newcommand{\Meet}{\ensuremath{\bigwedge}}
\renewcommand{\Join}{\ensuremath{\bigvee}}
\newcommand{\nb}[1]{\ensuremath{\#\mathrm{Blocks}(#1)}}
%\newcommand{\nb}[1]{\ensuremath{\#{#1}}}

%-------------------------------------------------------
\begin{document}
\title{On a problem of Palfy and Saxl}
\author{William DeMeo}
\date{November 13, 2013}

\maketitle

\section{Introduction}
In the paper \cite{PalfySaxl}, Peter Palfy and Jan Saxl pose the following 
\begin{quote}
  {\bf PROBLEM.} Let $\bA$ be a finite algebra with $\Con \bA \cong M_n$, $n\geq
  4$. If three nontrivial congruences of $\bA$ pairwise permute, does it follow
  that every pair of congruences of $\bA$ permute?
\end{quote}

These notes collect some notation and facts that might be useful for attacking this problem.
Throughout, $X$ denotes a finite set, $\Eq(X)$ denotes the lattice of
equivalence relations on $X$ and, for $\alpha \in \Eq(X)$ and $x\in X$, we
denote by $x/\alpha$ the equivalence class of $\alpha$ containing $x$.
We often refer to equivalnce classes as
``blocks,'' and we denote by $\nb{\alpha}$ the number of blocks of the
equivalence relation $\alpha$. 

For a given $\alpha\in \Eq(X)$ the map 
 $\phi_\alpha: x \mapsto x/\alpha$ is a
function from $X$ into the power set $\sP(X)$
with kernel $\ker \phi_\alpha = \alpha$. 
The \emph{block-size function} $x \mapsto |x/\alpha|$ is a function from $X$ into $\{1,2,\dots, |X|\}$.

We will often abuse notation and equate an equivalence relation with the
corresponding partition of the set $X$.  For example, we will equate 
the relation 
\[
\alpha = \{(0,0), (1,1), (2,2), (3,3), (0,1), (1,0), (2,3), (3,2)\}
\]
with the partition $|0,1|2,3|$, and often we resort to writing $\alpha = |0,1|2,3|$.

We say that $\alpha$ has \emph{uniform blocks} if 
all blocks of  $\alpha$ have the same size; or, equivalently,
the block-size function is constant: for all 
$x, y \in X$, $|x/\alpha| = |y/\alpha|$.  We will use $|x/\alpha|$, without specifying
a particular $x\in X$, to denote this block size.\footnote{Alternatively, we
  might consider using 
  $|x_{\cdot}/\alpha|$ to emphasize that every $x\in X$ can be substituted for
  $x_\cdot$ without changing the value of $|x_\cdot/\alpha|$, but this notation
  may be too cumbersome.}  
Thus, when $\alpha$
has uniform blocks, we have $|X|= |x/\alpha| \cdot \nb{\alpha}$.

%% If $\alpha$ has uniform blocks of size $r$, then the number of blocks 
%% of $\alpha$ is $m = |X|/r$.  
We say that two equivalence relations with uniform
blocks  have 
%% If $\beta$ is another equivalence relation, we say
%% that $\alpha$ and $\beta$ have 
\emph{complementary uniform block structure}, or simply \emph{complementary
  blocks}, if the number of blocks of one is equal to
the block size of the other. In other words, if $\alpha$ and $\beta$ are two
equivalence relations on $X$ with uniform block sizes $|x/\alpha|$ and
$|x/\beta|$, respectively, then $\alpha$ and $\beta$ have complementary blocks
if and only if $(\forall x)(\forall y)\, |x/\alpha|\cdot |y/\beta| = |X|$.

%% If $\alpha \in \Eq(X)$ has uniform blocks of size $r$, we say that $\alpha$ has
%% \emph{balanced uniform block structure}, or simply \emph{balanced blocks},
%% provided $r = |X|^{1/2}$.

Given two equivalence relations $\alpha$ and $\beta$ on $X$, the relation
\[
\alpha \circ \beta = \{(x,y) \in X^2: (\exists z)x\rel{\alpha} z
\rel{\beta} y\}
\]
is called the \emph{composition of $\alpha$ and $\beta$}, and if 
$\alpha \circ \beta = \beta \circ \alpha$ then $\alpha$ and $\beta$ are said to
\emph{permute}, or to be \emph{permuting} equivalence relations.  
Note that $\alpha \circ \beta \subseteq \alpha \join \beta$ with equility if and
only if $\alpha$ and $\beta$ permute.

The largest and smallest equivalence relations on $X$ are given by $1_X = X^2$
and $0_X = \{(x,x) : x \in X\}$, respectively.

%% denote the relation $(\forall x)(\forall y)(x,y) \in 0_X \leftrightarrow x=y$.
We say that $\alpha$ and $\beta$ are \emph{complementary} equivalence relations
on $X$ provided $\alpha \join \beta = 1_X$ and $\alpha \meet \beta = 0_X$.
\begin{lemma}
\label{lem:1}
Suppose $\alpha$ and $\beta$ are complementary equivalence relations on
$X$. Then $\alpha$ and $\beta$ permute if and only if they have complementary
blocks.  That is,
\[
\alpha \circ \beta \quad \Longleftrightarrow \quad (\forall x)(\forall y)\,
|x/\alpha|\cdot |y/\alpha| = |X|.
\]
\end{lemma}
\begin{corollary}
\label{cor:1}
Suppose $\alpha_1$, $\alpha_2$, $\alpha_3$ are pairwise complementary
equivalence relations on $X$. Then  $\alpha_1$, $\alpha_2$, $\alpha_3$ are
pairwise permuting if and only if they all have uniform blocks of size 
$\sqrt{|X|}$.  In other words,
\[
(\forall i)(\forall j) \, (i\neq j \longrightarrow \alpha_i \circ \alpha_j = 1_X)
\quad \Longleftrightarrow \quad (\forall i)(\forall x) \, |x/\alpha_i| =
\sqrt{|X|}.
\]
In this case, we clearly have $|x/\alpha_i| =\nb{\alpha_i}$. 
\end{corollary}
\begin{proof}[Proof of Lemma~\ref{lem:1}]
  Assume $\alpha \circ \beta = \alpha \join \beta = 1_X$. Then, for all $x\in X$
  we have
  \begin{equation}
    \label{eq:1}
%x/(\alpha\circ \beta) = \dot{\bigcup}_{y \in x/\alpha} y/\beta = X,
x/(\alpha\circ \beta) = \coprod_{y \in x/\alpha} y/\beta = X,
  \end{equation}
where $\coprod$ denotes disjoint union.  The union is disjoint since
$\alpha \meet \beta = 0_X$.  Since the union in~(\ref{eq:1}) is all of $X$,
every block of $\beta$ must appear in the union, so the block $x/\alpha$ has
exactly $\nb{\beta}$ elements. Since $x$ was arbitrary, $\alpha$ has uniform
blocks of size $|x/\alpha| =\nb{\beta}$. Similarly, 
$x/(\beta \circ \alpha) = \coprod_{y \in x/\beta} y/\alpha = X$, so 
$|x/\beta| =\nb{\alpha}$ holds for all $x\in X$.  Therefore, 
for all $x, y\in X$, we have
\[
|x/\alpha|\cdot |y/\beta| 
 = |x/\alpha|\cdot \nb{\alpha} = |X|.
\]

To prove the converse, suppse $\alpha$ and $\beta$ are pairwise complements
with complementary blocks.  Then $|x/\alpha|\cdot |y/\beta| = |X|$, 
thus $|y/\beta| = |x/\alpha|^{-1} \cdot |X|  = \nb{\alpha}$ 
hold for all $x, y \in X$.  Therefore, for all $x\in X$, we have
\begin{align*}
\bigl|x/(\alpha\circ \beta)\bigr| &= \bigl|\coprod_{y\in x/\alpha} y/\beta\bigr|
= \sum_{y \in x/\alpha} |y/\beta|\\
&= \sum_{y \in x/\alpha} \nb{\alpha} \\
&= |x/\alpha|\nb{\alpha} = |X|.
\end{align*}
This proves that $\alpha\circ \beta = 1_X$.
\end{proof}

\begin{proof}[Proof of Corollary~\ref{cor:1}]

\end{proof}

%% \begin{lemma}
%% Let $\{\alpha_i : 0\leq i < r\}$ be a set of pairwise complementary
%% equivalence relations on $X$.
%% \begin{enumerate}
%% \item If $\alpha_1 \circ \alpha_2 = \alpha_2 \circ \alpha_1$, then 
%% $\alpha_1$ and $\alpha_2$ have uniform blocks.
%% \item If $\alpha_1$ and $\alpha_2$ have uniform blocks of size $|X|^{1/2}$, then 
%% $\alpha_1 \circ \alpha_2 =  \alpha_2 \circ \alpha_1$.
%% \item Three pairwise complementary equivalence relations are pairwise permuting if and only
%%   if all three have uniform blocks of size $|X|^{1/2}$.
%% \end{enumerate}
%% these three
%% relations have complementary structure, and each
%% $\alpha_i$ has block size $|X|^{1/2} = n$, for some positive integer
%% $n$.  Thus, the number of blocks of each $\alpha_i$ is $|X|^{1/2}$, and
%% $|X| = n^2$.
%% \end{lemma}


\bibliographystyle{plainurl}
\bibliography{wjd}

\end{document}
