%% Based on a TeXnicCenter-Template by Gyorgy SZEIDL.
%%%%%%%%%%%%%%%%%%%%%%%%%%%%%%%%%%%%%%%%%%%%%%%%%%%%%%%%%%%%%

%------------------------------------------------------------
%
\documentclass{amsart}
%
%----------------------------------------------------------
% This is a sample document for the AMS LaTeX Article Class
% Class options
%        -- Point size:  8pt, 9pt, 10pt (default), 11pt, 12pt
%        -- Paper size:  letterpaper(default), a4paper
%        -- Orientation: portrait(default), landscape
%        -- Print size:  oneside, twoside(default)
%        -- Quality:     final(default), draft
%        -- Title page:  notitlepage, titlepage(default)
%        -- Start chapter on left:
%                        openright(default), openany
%        -- Columns:     onecolumn(default), twocolumn
%        -- Omit extra math features:
%                        nomath
%        -- AMSfonts:    noamsfonts
%        -- PSAMSFonts  (fewer AMSfonts sizes):
%                        psamsfonts
%        -- Equation numbering:
%                        leqno(default), reqno (equation numbers are on the right side)
%        -- Equation centering:
%                        centertags(default), tbtags
%        -- Displayed equations (centered is the default):
%                        fleqn (equations start at the same distance from the right side)
%        -- Electronic journal:
%                        e-only
%------------------------------------------------------------
% For instance the command
%          \documentclass[a4paper,12pt,reqno]{amsart}
% ensures that the paper size is a4, fonts are typeset at the size 12p
% and the equation numbers are on the right side
%
\usepackage{amsmath}%
\usepackage{amsfonts}%
\usepackage{amssymb}%
%(wjd) added stmaryrd and enumerate packages
\usepackage{stmaryrd,enumerate}
\usepackage{graphicx}
\usepackage{comment}
%------------------------------------------------------------
% Theorem like environments
%
\theoremstyle{plain}
\newtheorem{theorem}{Theorem}
\theoremstyle{definition}
\newtheorem{acknowledgement}{Acknowledgement}
\newtheorem{algorithm}{Algorithm}
\newtheorem{axiom}{Axiom}
\newtheorem{case}{Case}
\newtheorem{claim}{Claim}
\newtheorem{conclusion}{Conclusion}
\newtheorem{condition}{Condition}
\newtheorem{conjecture}{Conjecture}
\newtheorem{corollary}{Corollary}
\newtheorem{criterion}{Criterion}
\newtheorem{definition}{Definition}
\newtheorem{example}{Example}
\newtheorem{exercise}{Exercise}
\newtheorem{lemma}{Lemma}
\newtheorem{fact}{Fact}
\newtheorem{notation}{Notation}
\newtheorem{problem}{Problem}
\newtheorem{proposition}{Proposition}
\newtheorem{remark}{Remark}
\newtheorem{solution}{Solution}
\newtheorem{summary}{Summary}
\newtheorem{tool}{Tool}
\theoremstyle{definition}
\newtheorem{Example}{Example}
\numberwithin{equation}{section}
%--------------------------------------------------------
%(wjd) A few useful macros 
\renewcommand{\phi}{\ensuremath{\varphi}}
\newcommand{\core}{\ensuremath{\operatorname{core}}}
\newcommand{\Sub}{\ensuremath{\operatorname{Sub}}}
\newcommand{\<}{\ensuremath{\langle}}
\renewcommand{\>}{\ensuremath{\rangle}}
\newcommand{\lb}{\ensuremath{\llbracket}}
\newcommand{\rb}{\ensuremath{\rrbracket}}
%(wjd) I think it's better to use \lb H, G \rb than [H,G], 
%      since single brackets typically denotes the commutator.
%      If you don't like this, uncomment the next two lines and
%      everything will change back to single brackets:
% \renewcommand{\lb}{\ensuremath{[}}
% \renewcommand{\rb}{\ensuremath{]}}

%
%(wjd) I prefer angled less or equal symbols, but you can
%      change these back by commenting the next 6 lines.
\renewcommand{\leq}{\ensuremath{\leqslant}}
\renewcommand{\nleq}{\ensuremath{\nleqslant}}
\renewcommand{\geq}{\ensuremath{\geqslant}}
\renewcommand{\lneq}{\ensuremath{\lneqslant}}
\renewcommand{\gneq}{\ensuremath{\gneqslant}}
\renewcommand{\ngeq}{\ensuremath{\ngeqslant}}
%
%(wjd) I prefer angled "normal or equal" symbol, but you can
%      change this back by commenting the next line.
\renewcommand{\unlhd}{\ensuremath{\trianglelefteqslant}}


\newcommand{\bA}{\ensuremath{\mathbf{A}}}
\newcommand{\Con}{\ensuremath{\operatorname{Con}}}
\newcommand{\Eq}{\ensuremath{\operatorname{Eq}}}
\newcommand{\rel}{\ensuremath{\mathrel}}
\newcommand{\meet}{\ensuremath{\wedge}}
\newcommand{\join}{\ensuremath{\vee}}
\newcommand{\Meet}{\ensuremath{\bigwedge}}
\renewcommand{\Join}{\ensuremath{\bigvee}}

%-------------------------------------------------------
\begin{document}
\title{On a problem of Palfy and Saxl}
\author{William DeMeo}
\date{November 13, 2013}

\maketitle

\section{Introduction}
In the paper \cite{PalfySaxl}, Peter Palfy and Jan Saxl pose the following 
\begin{quote}
  {\bf PROBLEM.} Let $\bA$ be a finite algebra with $\Con \bA \cong M_n$, $n\geq
  4$. If three nontrivial congruences of $\bA$ pairwise permute, does it follow
  that every pair of congruences of $\bA$ permute?
\end{quote}

These notes address answer this question affirmatively.

First, we establish some basic facts and notation.
Throughout, $X$ denotes a finite set, $\Eq(X)$ denotes the lattice of
equivalence relations on $X$ and, for $\alpha \in \Eq(X)$ and $x\in X$, we
denote by $x/\alpha$ the equivalence class of $\alpha$ containing $x$.  
We often refer to equivalnce classes as
``blocks,'' and we say that $\alpha$ has \emph{uniform blocks} if, for all 
$x, y \in X$, $|x/\alpha| = |y/\alpha|$.  In other words, all equivalence
classes of  $\alpha$ have the same size.  

If $\alpha$ has uniform blocks of size $r$, then the number of blocks 
of $\alpha$ is $m = |X|/r$.  We say that two equivalence relations with uniform
blocks  have 
%% If $\beta$ is another equivalence relation, we say
%% that $\alpha$ and $\beta$ have 
\emph{complementary uniform block structure}, or simply \emph{complementary
  blocks}, if the number of blocks of one is equal to
the block size of the other. In other words, if $\alpha$ and $\beta$ are two
equivalence relations on $X$ with uniform block sizes $r_\alpha$ and
$r_\beta$, respectively, then $\alpha$ and $\beta$ have complementary block
structure if and only if $r_\alpha r_\beta = |X|$.

%% If $\alpha \in \Eq(X)$ has uniform blocks of size $r$, we say that $\alpha$ has
%% \emph{balanced uniform block structure}, or simply \emph{balanced blocks},
%% provided $r = |X|^{1/2}$.

Given two equivalence relations $\alpha$ and $\beta$ on $X$, the relation
\[
\alpha \circ \beta = \{(x,y) \in X^2: (\exists z)x\rel{\alpha} z
\rel{\beta} y\}
\]
is called the \emph{composition of $\alpha$ and $\beta$}, and if 
$\alpha \circ \beta = \beta \circ \alpha$ then $\alpha$ and $\beta$ are said to
\emph{permute}, or to be \emph{permuting} equivalence relations.  
Note that $\alpha \circ \beta \subseteq \alpha \join \beta$ with equility if and
only if $\alpha$ and $\beta$ permute.

We denote the largest and smallest equivalence relations on $X$ by $1_X$ and
$0_X$, respectively.
That is, $1_X = X^2$ and 
$0_X = \{(x,x) : x \in X\}$.
%% denote the relation $(\forall x)(\forall y)(x,y) \in 0_X \leftrightarrow x=y$.
We say that $\alpha$ and $\beta$ are \emph{complementary} equivalence relations
on $X$ provided $\alpha \join \beta = 1_X$ and $\alpha \meet \beta = 0_X$.
\begin{lemma}
Let $\{\alpha_i\}$ be some pairwise complementary
equivalence relations on $X$.
\begin{enumerate}
\item If $\alpha_1 \circ \alpha_2 = \alpha_2 \circ \alpha_1$, then 
$\alpha_1$ and $\alpha_2$ have uniform blocks.
\item If $\alpha_1$ and $\alpha_2$ have uniform blocks of size $|X|^{1/2}$, then 
$\alpha_1 \circ \alpha_2 =  \alpha_2 \circ \alpha_1$.
\item Three pairwise complementary equivalence relations are pairwise permuting if and only
  if all three have uniform blocks of size $|X|^{1/2}$.
\end{enumerate}
%% these three
%% relations have complementary structure, and each
%% $\alpha_i$ has block size $|X|^{1/2} = n$, for some positive integer
%% $n$.  Thus, the number of blocks of each $\alpha_i$ is $|X|^{1/2}$, and
%% $|X| = n^2$.
\end{lemma}


\bibliographystyle{plainurl}
\bibliography{wjd}

\end{document}
