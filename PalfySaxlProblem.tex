\begin{filecontents*}{refs.bib}
@inproceedings{McKenzie:1983,
    AUTHOR = {McKenzie, Ralph},
     TITLE = {Finite forbidden lattices},
 BOOKTITLE = {Universal algebra and lattice theory ({P}uebla, 1982)},
    SERIES = {Lecture Notes in Math.},
    VOLUME = {1004},
     PAGES = {176--205},
 PUBLISHER = {Springer},
   ADDRESS = {Berlin},
      YEAR = {1983},
   MRCLASS = {06B05 (06B10 08A30 08B15)},
  MRNUMBER = {716183 (85b:06006)},
MRREVIEWER = {Ralph Freese},
DOI = {10.1007/BFb0063438}
}
@BOOK{alvi:1987,
    AUTHOR = {McKenzie, Ralph N. and McNulty, George F. and Taylor, Walter F.},
     TITLE = {Algebras, lattices, varieties. {V}ol. {I}},
 PUBLISHER = {Wadsworth \& Brooks/Cole},
   ADDRESS = {Monterey, CA},
      YEAR = {1987},
     PAGES = {xvi+361},
      ISBN = {0-534-07651-3},
   MRCLASS = {08-01 (06-01)},
  MRNUMBER = {883644 (88e:08001)},
MRREVIEWER = {Gudrun Kalmbach},
}
@article{overalgebras,
  author = 	 {William DeMeo},
  title = 	 {Expansions of finite algebras and their congruence lattices},
  journal = 	 {Algebra Universalis},
volume = {69},
year={2013},
note = {Available from: \href{https://github.com/williamdemeo/Overalgebras}{github.com/williamdemeo/Overalgebras}},
DOI={10.1007/s00012-013-0226-3},
pages={257--278}
}
@BOOK{HM:1988,
    AUTHOR = {Hobby, David and McKenzie, Ralph},
     TITLE = {The structure of finite algebras},
    SERIES = {Contemporary Mathematics},
    VOLUME = {76},
 PUBLISHER = {American Mathematical Society},
   ADDRESS = {Providence, RI},
      YEAR = {1988},
     PAGES = {xii+203},
      ISBN = {0-8218-5073-3},
   MRCLASS = {08A05 (03C05 08-02 08B05)},
  MRNUMBER = {958685 (89m:08001)},
MRREVIEWER = {Joel Berman},
note = {Available from: \href{http://math.hawaii.edu/~ralph/Classes/619/HobbyMcKenzie-FiniteAlgebras.pdf}{math.hawaii.edu}}
}		
@article {MR3076179,
    AUTHOR = {Kearnes, Keith A. and Kiss, Emil W.},
     TITLE = {The shape of congruence lattices},
   JOURNAL = {Mem. Amer. Math. Soc.},
  FJOURNAL = {Memoirs of the American Mathematical Society},
    VOLUME = {222},
      YEAR = {2013},
    NUMBER = {1046},
     PAGES = {viii+169},
      ISSN = {0065-9266},
      ISBN = {978-0-8218-8323-5},
   MRCLASS = {08B05 (08B10)},
  MRNUMBER = {3076179},
MRREVIEWER = {James B. Nation},
       DOI = {10.1090/S0065-9266-2012-00667-8},
       URL = {http://dx.doi.org/10.1090/S0065-9266-2012-00667-8},
}
@article {PalfySaxl,
    AUTHOR = {P{\'a}lfy, P. P. and Saxl, J.},
     TITLE = {Congruence lattices of finite algebras and factorizations of
              groups},
   JOURNAL = {Comm. Algebra},
  FJOURNAL = {Communications in Algebra},
    VOLUME = {18},
      YEAR = {1990},
    NUMBER = {9},
     PAGES = {2783--2790},
      ISSN = {0092-7872},
     CODEN = {COALDM},
   MRCLASS = {08A30 (20E32)},
  MRNUMBER = {1063342 (91h:08003)},
MRREVIEWER = {R. S. Pierce},
}
@article {Palfy:1980,
    AUTHOR = {P{\'a}lfy, P{\'e}ter P{\'a}l and Pudl{\'a}k, Pavel},
     TITLE = {Congruence lattices of finite algebras and intervals in
              subgroup lattices of finite groups},
   JOURNAL = {Algebra Universalis},
  FJOURNAL = {Algebra Universalis},
    VOLUME = {11},
      YEAR = {1980},
    NUMBER = {1},
     PAGES = {22--27},
      ISSN = {0002-5240},
   MRCLASS = {08A30 (06B15 08A60 20E15)},
  MRNUMBER = {593011 (82g:08003)},
MRREVIEWER = {Ralph Freese},
DOI = {10.1007/BF02483080}
}
\end{filecontents*}

\documentclass{amsart}
\usepackage{amscd,amssymb}  % amsthm, amsmath are included by default
\usepackage{amsfonts}%
\usepackage{mathrsfs}
%(wjd) added stmaryrd and enumerate packages
\usepackage{stmaryrd,enumerate}
\usepackage{latexsym,mathrsfs,ifthen}
\usepackage{mathtools}
\usepackage[mathcal]{euscript}
\usepackage[style = ieee, urldate = comp]{biblatex}
\usepackage[colorlinks=true,urlcolor=black,linkcolor=black,citecolor=black]{hyperref}
\usepackage{scalefnt}
\usepackage{tikz}
\usepackage{color}
\usepackage{graphicx}
\usepackage{comment}
%% \usepackage[margin=1in]{geometry}
\usepackage{bm}
\usepackage{proof-dashed}
\newrobustcmd*{\entails}{\vdash}

%
%----------------------------------------------------------
% AMS LaTeX Article Class options
%        -- Point size:  8pt, 9pt, 10pt (default), 11pt, 12pt
%        -- Paper size:  letterpaper(default), a4paper
%        -- Orientation: portrait(default), landscape
%        -- Print size:  oneside, twoside(default)
%        -- Quality:     final(default), draft
%        -- Title page:  notitlepage, titlepage(default)
%        -- Start chapter on left:
%                        openright(default), openany
%        -- Columns:     onecolumn(default), twocolumn
%        -- Omit extra math features:
%                        nomath
%        -- AMSfonts:    noamsfonts
%        -- PSAMSFonts  (fewer AMSfonts sizes):
%                        psamsfonts
%        -- Equation numbering:
%                        leqno(default), reqno (equation numbers are on the right side)
%        -- Equation centering:
%                        centertags(default), tbtags
%        -- Displayed equations (centered is the default):
%                        fleqn (equations start at the same distance from the right side)
%        -- Electronic journal:
%                        e-only
%------------------------------------------------------------
% For instance the command
%          \documentclass[a4paper,12pt,reqno]{amsart}
% ensures that the paper size is a4, fonts are typeset at the size 12p
% and the equation numbers are on the right side
%

\usepackage{xspace}
%% \usepackage[smaller,printonlyused]{acronym}%
\usepackage[smaller]{acronym}%

\acrodef{cib}[CIB]{commutative idempotent binar}
\acrodef{cubs}[CUBS]{compatible uniform block structure}
\newcommand{\cubs}{\acs{cubs}\xspace}
\acrodef{pppc}[PPPC]{pairwise-permuting pairwise-complements}
\newcommand{\pppc}{\acs{pppc}\xspace}
\newcommand\dotsize{1pt}
%% \newcommand{\q}[2]{\ensuremath{\langle #1, #2\rangle}}
\newcommand{\q}[2]{\ensuremath{#1 < #2}}
\newcommand{\qab}{\ensuremath{\q{\alpha}{\beta}}}
%% \newcommand{\resB}{\ensuremath{|_{_B}}}
\newcommand{\resB}{\ensuremath{|_B}}
%% \newcommand{\resBi}{\ensuremath{|_{_{B_i}}}}
\newcommand{\resBi}{\ensuremath{|_{B_i}}}
%% \newcommand{\resU}{\ensuremath{|_{_U}}}
\newcommand{\resU}{\ensuremath{|_U}}
%% \newcommand{\resV}{\ensuremath{|_{_V}}}
\newcommand{\resV}{\ensuremath{|_V}}
%% \newcommand{\res}[1]{\ensuremath{|_{_{#1}}}}
\newcommand{\res}[1]{\ensuremath{|_{#1}}}
\newcommand{\suchthat}{\ensuremath{\mid}}  % (new notation)

%%%%%%%%%%%%%%%%%%%%%%%%%%%%%%%%%%%%%%%%%%%%%%%%%%%%%%%%%%%%%%%%%%
%%                      ifthenelse                             %%
%%%%%%%%%%%%%%%%%%%%%%%%%%%%%%%%%%%%%%%%%%%%%%%%%%%%%%%%%%%%%%%%%%
%% Use the \todo command to make notes of things to fix. 
%% Apply italics or bold or caps for emphasis as needed, e.g.,
%%
%%     \todo{(wjd) FIX THE NEXT LINE!}
%%     \todo{(wjd) \emph{Revise the paragraph above}}
%%
   \newboolean{todos}
   \setboolean{todos}{true}  % set to true to include TODO statements
%   \setboolean{todos}{false}  % set to false to exclude TODO statements
%% INCLUDE TODO ITEMS by commenting out `\setboolean{todos}{false}`
%% OMIT TODO ITEMS by uncommenting `\setboolean{todos}{false}`:
%%   \setboolean{todos}{false}
%%
   \newcommand{\todo}[1]{\ifthenelse{\boolean{todos}}{%
       ~\vskip0.1mm\noindent {\bf To do:} #1\vskip2mm}{}}
%%
%% optionally include everything (or just important parts)
   \newboolean{long}
   \setboolean{long}{true}  % set to true to include everything
%   \setboolean{long}{false}  % set to false for the condensed version
%%
%%%%%%%%%%%%%%%%%%%%%%%%%%%%%%%%%%%%%%%%%%%%%%%%%%%%%%%%%%%%%%%%%

%------------------------------------------------------------
% Theorem like environments
%
\theoremstyle{plain}
\newtheorem{theorem}{Theorem}
\theoremstyle{definition}
\newtheorem{acknowledgement}{Acknowledgement}
\newtheorem{algorithm}{Algorithm}
\newtheorem{axiom}{Axiom}
\newtheorem{case}{Case}
\newtheorem{claim}{Claim}
\newtheorem{conclusion}{Conclusion}
\newtheorem{condition}{Condition}
\newtheorem{conjecture}{Conjecture}
\newtheorem{corollary}{Corollary}
\newtheorem{criterion}{Criterion}
\newtheorem{definition}{Definition}
\newtheorem{example}{Example}
\newtheorem{exercise}{Exercise}
\newtheorem{lemma}{Lemma}
\newtheorem*{lem}{Lemma}
\newtheorem{fact}{Fact}[section]
\newtheorem{notation}{Notation}
\newtheorem{problem}{Problem}
\newtheorem{proposition}{Proposition}
\newtheorem*{remark}{Remark}
\newtheorem*{remarks}{Remarks}
\newtheorem{solution}{Solution}
\newtheorem{summary}{Summary}
\newtheorem{tool}{Tool}
\theoremstyle{definition}
\newtheorem{Example}{Example}
\numberwithin{equation}{section}
%--------------------------------------------------------
%(wjd) A few useful macros 
\newcommand{\bx}{\ensuremath{\mathbf{x}}}
\newcommand{\by}{\ensuremath{\mathbf{y}}}
\newcommand{\bz}{\ensuremath{\mathbf{z}}}
\renewcommand{\phi}{\ensuremath{\varphi}}
\newcommand{\<}{\ensuremath{\langle}}
\renewcommand{\>}{\ensuremath{\rangle}}
\newcommand{\lb}{\ensuremath{\llbracket}}
\newcommand{\rb}{\ensuremath{\rrbracket}}
\newcommand{\bB}{\ensuremath{\mathbf{B}}}
\newcommand{\bc}{\ensuremath{\mathbf{c}}}
%(wjd) I think it's better to use \lb H, G \rb than [H,G], 
%      since single brackets typically denotes the commutator.
%      If you don't like this, uncomment the next two lines and
%      everything will change back to single brackets:
% \renewcommand{\lb}{\ensuremath{[}}
% \renewcommand{\rb}{\ensuremath{]}}

%
%(wjd) I prefer angled less or equal symbols, but you can
%      change these back by commenting the next 6 lines.
\renewcommand{\leq}{\ensuremath{\leqslant}}
\renewcommand{\nleq}{\ensuremath{\nleqslant}}
\renewcommand{\geq}{\ensuremath{\geqslant}}
\renewcommand{\lneq}{\ensuremath{\lneqslant}}
\renewcommand{\gneq}{\ensuremath{\gneqslant}}
\renewcommand{\ngeq}{\ensuremath{\ngeqslant}}
%
%(wjd) I prefer angled "normal or equal" symbol, but you can
%      change this back by commenting the next line.
\renewcommand{\unlhd}{\ensuremath{\trianglelefteqslant}}

\newcommand{\ba}{\ensuremath{\mathbf{a}}}
\newcommand{\bb}{\ensuremath{\mathbf{b}}}
\newcommand{\bt}{\ensuremath{\mathbf{t}}}
\newcommand{\bu}{\ensuremath{\mathbf{u}}}
\newcommand{\bv}{\ensuremath{\mathbf{v}}}
\newcommand{\bw}{\ensuremath{\mathbf{w}}}
\newcommand{\nn}{\ensuremath{\underline{n}}}
\newcommand{\mm}{\ensuremath{\underline{m}}}
\newcommand{\kk}{\ensuremath{\underline{k}}}

\newcommand{\defn}[1]{{\it #1}}
\newcommand{\Peter}{P\'{e}ter\xspace}
\newcommand{\Palfy}{P\'{a}lfy\xspace}
\newcommand{\Pudlak}{Pudl\'ak\xspace}
\newcommand{\PP}{P\'{a}lfy-Pudl\'{a}k\xspace}
\newcommand{\PAP}{P\'{a}lfy\ and Pudl\'{a}k\xspace}
\newcommand{\N}{\ensuremath{\mathbb{N}}}
\newcommand{\bA}{\ensuremath{\mathbf{A}}}
\newcommand{\bS}{\ensuremath{\mathbf{S}}}
\newcommand{\bT}{\ensuremath{\mathbf{T}}}
\newcommand{\sP}{\ensuremath{\mathscr{P}}}
\newcommand{\sS}{\ensuremath{\mathcal{S}}}
%\newcommand{\ker}{\ensuremath{\operatorname{ker}}}

\DeclareMathOperator{\typ}{typ}
\DeclareMathOperator{\id}{id}
\DeclareMathOperator{\Eq}{Eq}
\DeclareMathOperator{\Cg}{Cg}
\DeclareMathOperator{\Con}{Con}
\DeclareMathOperator{\Sub}{Sub}
\DeclareMathOperator{\Pol}{Pol}
\DeclareMathOperator{\Clo}{Clo}
\newcommand{\core}{\ensuremath{\operatorname{core}}}

\newcommand{\rel}{\ensuremath{\mathrel}}
\newcommand{\ralpha}{\ensuremath{\mathrel{\alpha}}}
\newcommand{\rbeta}{\ensuremath{\mathrel{\beta}}}
\newcommand{\rgamma}{\ensuremath{\mathrel{\gamma}}}
\newcommand{\rdelta}{\ensuremath{\mathrel{\tau}}}
\newcommand{\rrho}{\ensuremath{\mathrel{\tau}}}
\newcommand{\rtau}{\ensuremath{\mathrel{\tau}}}
\newcommand{\rtheta}{\ensuremath{\mathrel{\theta}}}
\newcommand{\sansC}{\ensuremath{\mathsf{C}}}
\newcommand{\hatmap}{\ensuremath{\widehat{\phantom{x}}}} %bd ok

\newcommand{\meet}{\ensuremath{\wedge}}
\newcommand{\join}{\ensuremath{\vee}}
\newcommand{\Meet}{\ensuremath{\bigwedge}}
\renewcommand{\Join}{\ensuremath{\bigvee}}
%% \newcommand{\nb}[1]{\ensuremath{\#\mathrm{Blocks}(#1)}}
%% \newcommand{\nb}[1]{\ensuremath{\operatorname{N}(#1)}}
%% \newcommand{\nb}[1]{\ensuremath{\boldsymbol{\nu}(#1)}}
%% \newcommand{\nb}[1]{\ensuremath{\operatorname{\boldsymbol{\nu}}(#1)}}
\newcommand{\nb}[1]{\ensuremath{|X/#1|}}
%% \newcommand{\nb}[1]{\ensuremath{\#{#1}}}

%-------------------------------------------------------
\addbibresource{refs.bib}

\begin{document}
\title{On a problem of P\'{a}lfy and Saxl}
\author{William DeMeo}
%\date{November 13, 2013}
%\date{14 Feb 2016}
\date{\today}

\maketitle
\begin{abstract}
Among the oldest open questions in universal algebra is the following: which finite lattices are
congruence lattices of finite algebras?  In this note we consider a related problem about a special class of
lattice, namely, the ``diamonds'' $M_n$, which are height-two modular lattice with $n$ atoms.
In \cite{PalfySaxl}, \Peter \Palfy and Jan Saxl used deep results from group theory to prove that if a finite
algebra has a congruence lattice isomorphic to $M_n$ ($n>3$), and at least three atoms
pairwise permute, then $n-1$ is a power of a prime number.
In the process, they also showed that if a $G$-set has an $M_n$ ($n>3$) congruence lattice and
at least 3 atoms pairwise permute, then all congruences permute.

\Palfy and Saxl leave open the following question: if a finite algebra has congruence
lattice $M_n$ ($n>3$), and if at least 3 atoms pairwise permute, does it follow that
all congruences permute? In this note, we present some of the background required understand 
\Palfy and Saxl's question, including some tame congruence theory that dispenses with an
``easy'' case of the question.  
%% (Specifically, if the algebra in question is not \emph{strongly abelian}, then it has a 
%% Malcev term operation, so all congruences permute.)  
We then discuss a strategy that we hope will lead to a solution in the harder case. 
%% lead to a solution in the harder (strongly abelian) case. 
\end{abstract}

\section{Introduction}
This paper mainly concerns lattices of the form displayed in Figure~\ref{fig:Mn}.
\begin{figure}[!h]
      \begin{tikzpicture}[scale=1.2]
      \node (bot) at (0,0)  [draw, circle, inner sep=\dotsize] {};
      \node (a1) at (-1,1)  [draw, circle, inner sep=\dotsize] {};
      \node (a2) at (-.5,1)  [draw, circle, inner sep=\dotsize] {};
      \node (a3) at (0,1)  [draw, circle, inner sep=\dotsize] {};
      \node (a4) at (.7,1) {}; % [draw, circle, inner sep=\dotsize] {};
      \node (b) at (1,1)  [draw, circle, inner sep=\dotsize] {};
      \node (top) at (0,2)  [draw, circle, inner sep=\dotsize] {};
      %% \draw (a1) node [left] {$\alpha_1$};
      %% \draw (a2) node [right] {$\alpha_2$};
      \draw (a4) node [left] {$\cdots$};
      %% \draw (b) node [right] {$\alpha_n$};
      %% \draw (bot) node [below] {$0_X$};
      %% \draw (top) node [above] {$1_X$};
      \draw[semithick] (bot)--(a1)--(top)--(a2)--(bot)--(a3)--(top)--(b)--(bot);
      %% \draw[semithick] (bot)--(a1)--(top)--(a2)--(bot)--(b)--(top);
      \end{tikzpicture}
  \caption{The lattice $M_n$ has $n$ ``atoms'' for a total of $n+2$ elements.}
  \label{fig:Mn}
\end{figure}
In the paper \cite{PalfySaxl}, \Peter \Palfy and Jan Saxl pose 
the following: 
\begin{quote}
  {\scshape Problem.}
  Let $\bA$ be a finite algebra with $\Con \bA \cong M_n$, $n\geq
  4$. If three nontrivial congruences of $\bA$ pairwise permute, does it follow
  that every pair of congruences of $\bA$ permute?
\end{quote}
In this note, we answer this question affirmatively for a special class of
algebras. 
Specifically, if the algebra in question is not \emph{strongly abelian}, then it has a 
Malcev term operation, so all congruences permute.  
We also give some examples to guide intuition, and suggest some 
possible strategies for handling the harder (strongly abelian) case. 

\subsection{Notation}
Throughout, $X$ denotes a finite set and $\Eq(X)$ denotes the lattice of
equivalence relations on $X$. For $\alpha \in \Eq(X)$ and $x\in X$, we
let $x/\alpha$ denote the equivalence class of $\alpha$ containing $x$, and 
$X/\alpha$ denotes the set of all equivalence classes of $\alpha$. That is,
\[
x/\alpha = \{y\in X : x \mathrel{ \alpha } y\} \quad \text{ and } \quad
X/\alpha = \{x/\alpha : x\in X \}.
\]
The largest and smallest equivalence relations on $X$ are denoted by $1_X = X^2$
and $0_X = \{(x,x) : x \in X\}$, respectively.
We say that $\alpha$ and $\beta$ are \emph{complementary} equivalence relations
on $X$ provided $\alpha \join \beta = 1_X$ and $\alpha \meet \beta = 0_X$.


We often refer to equivalence classes as
``blocks.''  
For a given $\alpha\in \Eq(X)$, the number of blocks 
of $\alpha$ is $|X/\alpha|$. The map $\phi_\alpha: x \mapsto x/\alpha$ is a
function from $X$ into the power set $\sP(X)$
with kernel $\ker \phi_\alpha = \alpha$.
The \emph{block-size function} $x \mapsto |x/\alpha|$ is a function from $X$ 
into $\{1,2,\dots, |X|\}$. 

We will often abuse notation and identify an equivalence relation with the
corresponding partition of the set $X$.  For example, if $X = \{0,1,2,3,4\}$, 
we identify the relation 
\[
\alpha = \{(0,0), (1,1), (2,2), (3,3), (4,4), (0,1), (1,0), (2,3), (3,2)\}% \subseteq X^2
\]
with the partition $|0,1|2,3|4|$, and will even write $\alpha = |0,1|2,3|4|$. 
Using this example to illustrate the notation above, we have
$\varphi_\alpha(0) = 0/\alpha = \{0,1\} = 1/\alpha = \varphi_\alpha(1)$.
Similarly, $\varphi_\alpha(2) = \{2,3\}= \varphi_\alpha(3)$, and 
$\varphi_\alpha(4) = 4/\alpha = \{4\}$.  Thus, we have $X/\alpha = \{\{0,1\}, \{2,3\}, \{4\}\}$, so
$|X/\alpha| = 3$ in this case.

We say that $\alpha$ has \emph{uniform blocks} if all blocks of  $\alpha$ have
the same size; or, equivalently, the block-size function is constant: for all 
$x, y \in X$, $|x/\alpha| = |y/\alpha|$.  
In this case we will use $|x/\alpha|$ without specifying
a particular $x$ to denote this block size.
%% \footnote{Alternatively, we
%%   might consider using 
%%   $|x_{\cdot}/\alpha|$ to emphasize that every $x\in X$ can be substituted for
%%   $x_\cdot$ without changing the value of $|x_\cdot/\alpha|$, but this notation
%%   may be too cumbersome.}  
Thus, if $\alpha$
has uniform blocks, then
\[
|X|= |x/\alpha| \,|X/\alpha| %% = |x/\alpha| \nb{\alpha}
\quad \text{(for all $x\in X$).}
\]
We say that two equivalence relations with uniform
blocks  have 
\acfi{cubs} if the number of blocks of one is equal to
the block size of the other. That is,
$\alpha$ and $\beta$ 
have \cubs
iff $|x/\alpha||y/\beta| = |X|$ (for all $x$ and $y$).

%% If $\alpha \in \Eq(X)$ has uniform blocks of size $r$, we say that $\alpha$ has
%% \emph{balanced uniform block structure}, or simply \emph{balanced blocks},
%% provided $r = |X|^{1/2}$.

If $\alpha$ and $\beta$ are binary relations on $X$, then the relation
\begin{equation}
  \label{eq:7}
\alpha \circ \beta = \{(x,y) \in X^2: \exists z\,.\, x \ralpha z \rbeta y\}
\end{equation}
is called the \emph{composition of $\alpha$ and $\beta$}. 
Obviously, $(x,y) \in \alpha\circ \beta$ iff $(y,x) \in \beta\circ \alpha$.
If $\alpha \circ \beta = \beta \circ \alpha$, then we call $\alpha$ and $\beta$ 
\emph{permuting} relations and we say that $\alpha$ and $\beta$ \emph{permute}. 

In this paper, we are mostly concerned with the set 
$\Eq(X)$ of equivalence relations on $X$. 
$\Eq(X)$ is a lattice if we define, 
for $\alpha, \beta \in \Eq(X)$, the meet $\alpha\meet \beta$
to be the usual intersection of sets $\alpha$ and $\beta$ and the join 
$\alpha \join \beta$ to be the equivalence relation generated by
$\alpha$ and $\beta$, that is, the smallest equivalence relation containing both 
$\alpha$ and $\beta$.  It is not too difficult to prove that 
$\alpha \join \beta$ is equal to 
\[
\alpha \join \beta = (\alpha\circ \beta) \cup 
(\alpha\circ \beta \circ \alpha) \cup
(\alpha\circ \beta \circ \alpha \circ \beta) \cup \cdots
\]
and that 
$\alpha \circ \beta \subseteq \alpha \join \beta$,
with equality if and only if $\alpha$ and $\beta$ permute.
Note that, if $\alpha$ and $\beta$ do not permute, then 
the relation $\alpha\circ \beta$ is not symmetric, and therefore 
is not an equivalence relation.
Nonetheless, we will abuse the notation
$x/\alpha\circ \beta$ in such cases, and define
\begin{equation*}
  %% \label{eq:10}
x/\alpha\circ \beta:= \{y\in X : \exists z\,.\, x \ralpha z \rbeta y\}
\;\text{ and }\;
\alpha\circ \beta\backslash x:= \{y\in X : \exists z\,.\, y \ralpha z \rbeta x\}.
\end{equation*}
Although these are not necessarily
equivalence classes, they are well-defined subsets of $X$. 
%% the two sets in~(\ref{eq:10}) are equal.
From the definition (\ref{eq:7}) it is clear that %for every $x\in X$ 
$(x,y) \in \alpha\circ \beta$ if and only if
$y \in z/\beta$ for some $z\in x/\alpha$.  Thus, for every $x\in X$,
  \begin{equation}
    \label{eq:1}
x/\alpha\circ \beta = \bigcup_{z \in x/\alpha} z/\beta.
%x/(\alpha\circ \beta) = \coprod_{y \in x/\alpha} y/\beta 
  \end{equation}
 Furthermore,
$\alpha$ and $\beta$ are permuting equivalence relations 
iff for each $x\in X$ we have $x/\alpha\circ \beta = \alpha\circ \beta\backslash x$.

We conclude this subsection with one more definition
that is nonstandard but is quite useful for our application.
If $\Gamma$ is a set of equivalence relations, 
we say that $\Gamma$ consists of 
\acfi{pppc}
%\pppc %{\it \ac{PPPC}}
%% {\it \acl{PPPC}} 
if the following conditions hold for all $\gamma \neq \delta$ in $\Gamma$:
\begin{enumerate}[(i)]
\item $\gamma \join \delta = 1_X$;
\item $\gamma \meet \delta = 0_X$;  
\item $\gamma \circ \delta = \delta \circ \gamma$.  
\end{enumerate}




\subsection{Basic observations}
%% denote the relation $(\forall x)(\forall y)(x,y) \in 0_X \leftrightarrow x=y$.
\begin{lemma}
\label{lem:1}
Suppose $\alpha$ and $\beta$ are complementary equivalence relations on
$X$. Then $\alpha$ and $\beta$ permute if and only if they have \cubs.
That is,
\begin{equation}
  \label{eq:9}
\alpha \circ \beta =1_X \quad \Longleftrightarrow \quad (\forall x)(\forall z)\;
|x/\alpha| |z/\beta| = |X|.
\end{equation}
\end{lemma}
\begin{corollary}
\label{cor:1}
Suppose $\alpha_1$, $\alpha_2$, $\alpha_3$ are pairwise complementary
equivalence relations on the finite set $X$. 
Then  $\alpha_1$, $\alpha_2$, $\alpha_3$ pairwise permute if and only if they
have uniform blocks of size $\sqrt{|X|}$.  That is,
\[
(\forall i)(\forall j) \, (i\neq j \longrightarrow \alpha_i \circ \alpha_j = 1_X)
\quad \Longleftrightarrow \quad (\forall i)(\forall x) \, |x/\alpha_i| =
\sqrt{|X|}.
\]
In this case,  $|x/\alpha_i| = |X/\alpha_i|$. %\nb{\alpha_i}$. 
\end{corollary}
\begin{proof}[Proof of Lemma~\ref{lem:1}]
Suppose $\alpha$ and $\beta$ are complementary
equivalence relations.  Then, since $\alpha \meet \beta = 0_X$,
the union in~(\ref{eq:1}) is disjoint; we denote this by writing
  \begin{equation}
    \label{eq:100}
%x/(\alpha\circ \beta) = \dot{\bigcup}_{y \in x/\alpha} y/\beta = X,
x/(\alpha\circ \beta) = \coprod_{z \in x/\alpha} z/\beta.
  \end{equation}
%% where $\coprod$ denotes disjoint union. 
Also, since 
$\alpha \circ \beta = \alpha \join \beta = 1_X$, we have 
$x/(\alpha\circ \beta) =  X$ for 
every $x\in X$.
Thus the union in~(\ref{eq:100}) is all of $X$,
so every block of $\beta$ appears in this union.
It follows that the size of the block $x/\alpha$ is exactly  
the number $|X/\beta|$ of blocks of $\beta$. 
That is, $|x/\alpha| =|X/\beta|$.
%$\nb{\beta}$. 
As $x$ was arbitrary, $\alpha$ has uniform
blocks of size %% $|x/\alpha| =
$|X/\beta|$. %$\nb{\beta}$. 
Observe that, since $\alpha$ has uniform blocks, we have
\begin{equation}
\label{eq:8}
|x/\alpha|\, |X/\alpha| = |X|.  
\end{equation}

The same argument with the roles of 
$\alpha$ and $\beta$ swapped gives
%% $x/(\beta \circ \alpha) = \coprod_{z \in x/\beta} z/\alpha = X$, 
%% and
$|x/\beta| = |X/\alpha|$. %$\nb{\beta}$. 
Thus, for all $x, z\in X$ we have
\[
|x/\alpha|\, |z/\beta| 
 = |x/\alpha| \,|X/\alpha|
 %% = |X/\beta|\, |z/\beta|
 = |X/\beta|\, |X/\alpha|.
\]
This and (\ref{eq:8}) imply
$|x/\alpha|\, |z/\beta| = |X|$, as desired.

To prove the converse, suppose $\alpha$ and $\beta$ have \cubs.
%% are pairwise complements with complementary blocks. 
Then $|x/\alpha|\, |y/\beta| = |X|$, 
so $|y/\beta| = |x/\alpha|^{-1}  |X|  =  |X/\alpha|$, for all 
$x, y\in X$. 
Therefore, 
%% \begin{align*}
%% \bigl|x/(\alpha\circ \beta)\bigr| &= \bigl|\coprod_{y\in x/\alpha} y/\beta\bigr|
%% = \sum_{y \in x/\alpha} |y/\beta|= \sum_{y \in x/\alpha} \, |X/\alpha| \\
%% &= |x/\alpha| \, |X/\alpha| = |X|.
%% \end{align*}
\[
\bigl|x/(\alpha\circ \beta)\bigr| = \bigl|\coprod_{y\in x/\alpha} y/\beta\bigr|
= \sum_{y \in x/\alpha} |y/\beta|= \sum_{y \in x/\alpha} \, |X/\alpha| = |x/\alpha| \, |X/\alpha| = |X|,
\]
for all $x\in X$. This proves $\alpha\circ \beta = 1_X$, as desired.
\end{proof}

\begin{proof}[Proof of Cor.~\ref{cor:1}]
Let $\alpha_1, \alpha_2, \alpha_3$ be pairwise complementary equivalence relations.

\noindent ($\Rightarrow$)
Assume $\alpha_1, \alpha_2, \alpha_3$ are pairwise permuting (hence, \pppc). Then, by Lemma~\ref{lem:1},
\begin{align*}
|x/\alpha_1|\, |x/\alpha_2| &= |X|\\
|x/\alpha_1|\, |x/\alpha_3| &= |X|\\
|x/\alpha_2|\, |x/\alpha_3| &= |X|.
\end{align*}
By the third of these equations, we have 
$|x/\alpha_3| = |x/\alpha_2|^{-1} |X|$.
Substituting this into the second equation gives 
$|x/\alpha_1|\, |x/\alpha_2|^{-1} |X| = |X|$,
or $|x/\alpha_2|= |x/\alpha_1|$.
Substituting this into the first equation gives 
$|x/\alpha_1|= \sqrt{|X|}$. This can finally be substituted
%% Permuting the
%% roles of $\alpha_1$, $\alpha_2$, $\alpha_3$, 
%% the same argument yields 
back into the second and third
equations to arrive at $|x/\alpha_2|= \sqrt{|X|} = |x/\alpha_3|$.

%% Since $\alpha_1$ and $\alpha_2$ permute and are complements, Lemma~\ref{lem:1}
%% implies they have complementary uniform block structure (\cubs), so
%% \begin{equation}
%%   \label{eq:2}
%% |x/\alpha_1| = |x/\alpha_2|^{-1}  |X|=  |X/\alpha_2|.
%% \end{equation}
%% Similarly, since $\alpha_1$ and $\alpha_3$ permute, we have
%% $|x/\alpha_1| = |x/\alpha_3|^{-1}  |X| = |X/\alpha_3|$.
%% Therefore, $|X/\alpha_2| = |X/\alpha_3|$.  
%% Since $\alpha_2$ and $\alpha_3$ permute, we have
%% \begin{equation}
%%   \label{eq:3}
%% |x/\alpha_2| = |x/\alpha_3|^{-1}  |X| = |X/\alpha_3|,
%% \end{equation}
%% and the latter is equal to $|X/\alpha_2|$.  Therefore,
%% $|X| = |x/\alpha_2|\, |X/\alpha_2| = |x/\alpha_2| \, |x/\alpha_2|$, so 
%% $|x/\alpha_2|  = \sqrt{|X|}$. By (\ref{eq:2}) and (\ref{eq:3}), it follows
%% that
%% $|x/\alpha_i| = \sqrt{|X|} = |X/\alpha_i|$ for $i = 1, 2, 3$.

\noindent $(\Leftarrow)$ The right-to-left direction follows immediately from 
the right-to-left direction in (\ref{eq:9}).
%%  since, if $\alpha_i$ and $\alpha_j$ are
%% complementary equivalence relations on $X$ with 
%% $|x/\alpha_i| = \sqrt{|X|}$, then 
%% $|X/\alpha_i| = \sqrt{|X|}$, so
%% $\alpha_i \circ \alpha_j = 1_X$.
\end{proof}

Define the set $\sS\subseteq \Eq(X)$ as follows:
\[
\sS = \{ \alpha \in \Eq(X) : (\forall x) \; |x/\alpha|  
= \sqrt{|X|} = |X/\alpha|\}.
\]
From Corollary~\ref{cor:1} we see that the \Palfy-Saxl problem can be rephrased
as follows:

\medskip

\begin{quote}
  {\sc Problem.} Let $\bA$ be a finite algebra with $\Con \bA \cong M_n$, $n\geq
  4$. %If three nontrivial congruences of $\bA$ pairwise permute, does it follow
  If the set $\sS$ contains three atoms of
  $\Con \bA$,
  %% have Property~(\ref{eq:4}) below
  does it follow that $\sS$ contains every atom of 
  $\Con \bA$?
  %% has Property~(\ref{eq:4})?
  %%   \begin{equation}
  %%     \label{eq:4}
  %% (\forall x) \; |x/\alpha|  = \sqrt{|X|} = \nb{\alpha}
  %%   \end{equation}
\end{quote}

\medskip

%% \noindent To prove that the answer is ``no,'' it would suffice to find a 
%% congruence lattice $\Con \bA \cong M_n$, $n\geq 4$, where 3 atoms have
%% Property~(\ref{eq:4})   
%% % $|x/\alpha| = \sqrt{|X|}$
%%  and one atom $\beta$ has $|x/\beta| < \sqrt{|X|} < \nb{\beta}$.
%% \bigskip

\noindent To prove that the answer is ``yes,'' it suffices to show that 
whenever $M_n \cong L \leq \Eq(X)$ has 3 atoms in $\sS$ and
%% with Property~(\ref{eq:4}) and
an atom $\beta$ with $|x/\beta| < \sqrt{|X|}$, then $L$ is not a
congruence lattice.  
%% For example, we might find a method which shows in
%% such cases that the closure of the lattice will always contain some 
%% $\beta < \theta < 1_X$.

\subsection{Clones and tolerances}
Let $\bA =  \<A, \dots \>$ be an algebra with congruence lattice $\Con\<A, \dots \>$.
Recall that a \emph{clone} on a non-void set $A$ is a set of operations on $A$
that contains the projection operations and is closed under compositions.
The \emph{clone of term operations} of the algebra $\bA$, denoted by $\Clo \bA$,
is the smallest clone on $A$ containing the basic operations of $\bA$.
The \emph{clone of polynomial operations} of $\bA$, denoted by
$\Pol \bA$, is the clone generated by the basic operations
of $\bA$ and the constant unary maps on $A$. The set of $n$-ary members of
$\Pol \bA$ is denoted by $\Pol_n \bA$.
It often simplifies matters to shift our attention away from the 
basic operations of $\bA$ and focus instead on $\Clo\bA$ or $\Pol\bA$, or even 
$\Pol_1\bA$.  If our primary interest is the congruence lattice $\Con \bA$,
this is not a limitation because of the following fact:
$\Con\<A, \dots \>  = \Con \<A, \Clo \bA\> = \Con \<A, \Pol \bA\>=
\Con \<A, \Pol_1 \bA\>$. 
For proof, see e.g.~\cite[Theorem~4.18]{alvi:1987}.

A reflexive, symmetric, compatible binary relation $T\subseteq A^2$ is called a 
\defn{tolerance of $\bA$}.  
Any congruence relation of an algebra is a tolerance, and the 
transitive closure of any tolerance is a congruence relation.
We call a tolerance of $\bA$ \defn{connected} if its transitive closure is all of $A^2$.
As a compatible and binary relation, 
a tolerance induces a subalgebra of $\bA^2$, so we often denote a
tolerances with boldface letters, as in $\bT \leq \bA^2$.  If we have a pair 
$(\bu, \bv) \in A^m\times A^m$ of $m$-tuples of $A$, then we write  
$\bu \mathrel{\underline{\bT}} \bv$ just in case $\bu(i) \mathrel{\bT} \bv(i)$ for all $i\in \mm$. 

\subsection{Centralizers and Abelian Algebras}
We will need some basic facts about abelian algebras. Let $\bA = \<A, F^{\bA}\>$ be an algebra.
%% and let $\Clo\bA$ denote the clone of term operations of $\bA$; that is, 
%% $\Clo\bA$ is the smallest set of operations on $\bA$ that includes the basic operations of 
%% $\bA$ and the projections, and that is closed under general composition.
Suppose $\bS$ and $\bT$ are tolerances on $\bA$.  An \defn{$\bS,\bT$-matrix} 
is a $2\times 2$ array of the form
\[
\begin{bmatrix*}[r] t(\ba,\bu) & t(\ba,\bv)\\ t(\bb,\bu)&t(\bb,\bv)\end{bmatrix*},
\]
where $t$, $\ba$, $\bb$, $\bu$, $\bv$ have the following properties:
\begin{enumerate}[\rm(i)]
\item $t\in \Clo_{\ell + m}\bA$,
\item $(\ba, \bb)\in A^\ell\times A^\ell$ and $\ba \mathrel{\underline{\bS}} \bb$,
\item $(\bu, \bv)\in A^m\times A^m$ and $\bu \mathrel{\underline{\bT}} \bv$.
\end{enumerate}
Let $\delta$ be a congruence relation of $\bA$.
If the entries of every $\bS,\bT$-matrix satisfy
\begin{equation}
  \label{eq:22}
t(\ba,\bu) \mathrel{\delta} t(\ba,\bv)\quad \iff \quad t(\bb,\bu) \mathrel{\delta} t(\bb,\bv),
\end{equation}
then we say that $\bS$ \defn{centralizes $\bT$ modulo} $\delta$ and we write 
$\sansC(\bS, \bT; \delta)$.
That is, $\sansC(\bS, \bT; \delta)$ holds iff 
(\ref{eq:22}) holds \emph{for all}
$\ell$, $m$, $t$, $\ba$, $\bb$, $\bu$, $\bv$ satisfying properties (i)--(iii).
The condition $\sansC(\bS, \bT; 0_{\bA})$ is sometimes called the 
\defn{$\bS, \bT$-term condition}, and when it holds we say  that
$\bS$ \defn{centralizes} $\bT$, and write
$\sansC(\bS, \bT)$.
A tolerance $\bT$ is called \defn{abelian} if $\sansC(\bT, \bT)$.
An algebra $\bA$ is called \defn{abelian} if $1_\bA$ is abelian.

\begin{remark}
An algebra $\bA$ is abelian iff $\sansC(1_\bA, 1_\bA)$ iff
\ifthenelse{\boolean{long}}{%
\[
\forall \ell \in \{0,1,2,\dots \}, 
\quad \forall m \in  \{1,2,\dots \},
\quad \forall t\in \Clo_{\ell + m}\bA,
\quad \forall (a, b)\in A^\ell\times A^\ell,
\]
\[
\ker t(a, \cdot)=\ker t(b, \cdot).
\]}{
\[
\forall \ell, m \in \N,
%% \quad \forall m \in  \{1,2,\dots \},\]
\quad \forall t\in \Clo_{\ell + m}\bA,
\quad \forall (\ba, \bb)\in A^\ell\times A^\ell,
\]
\[
\ker t(\ba, \cdot)=\ker t(\bb, \cdot).
\]}
\end{remark}

\subsection{Facts}
We now collect some useful facts about centralizers of congruence relations.
These facts are well-known and easy to prove. (See, for example,~\cite{MR3076179}.)
\begin{lemma}
\label{lem:centralizers}
Let $\bA$ be an algebra with congruences 
$\alpha, \beta, \gamma, \alpha', \beta' \in \Con(\bA)$, and let 
$\bB$ be a subalgebra of $\bA$. Then,
\begin{enumerate}
\item \label{fact:centralizing_over_meet}
  $\sansC(\alpha, \beta; \alpha \meet \beta)$;
\item \label{fact:centralizing_over_meet2}
  if $\sansC(\alpha, \beta; \gamma)$ and $\sansC(\alpha, \beta; \gamma')$, then
  $\sansC(\alpha, \beta; \gamma \meet \gamma')$;
\item \label{fact:centralizing_over_join1}
  if $\sansC(\alpha, \beta; \gamma)$ and $\sansC(\alpha', \beta; \gamma)$, then
  $\sansC(\alpha \join \alpha', \beta; \gamma)$;
%% \item \label{fact:centralizing_over_join2}
%%   if $\sansC(\alpha, \beta; \gamma)$ and $\sansC(\alpha, \beta'; \gamma)$, then
%%   $\sansC(\alpha, \beta \join \beta'; \gamma)$;
\item \label{fact:monotone_centralizers1}
  if $\sansC(\alpha, \beta; \gamma)$ and $\alpha' \leq \alpha$, then 
  $\sansC(\alpha', \beta; \gamma)$;
\item \label{fact:monotone_centralizers2}
  if $\sansC(\alpha, \beta; \gamma)$ and $\beta' \leq \beta$, then
  $\sansC(\alpha, \beta'; \gamma)$;
\item \label{fact:dual_monotone_centralizers}
  if $\sansC(\alpha, \beta; \gamma)$ and $\gamma \leq \gamma'$, then
  $\sansC(\alpha, \beta; \gamma')$;
\item \label{item:subalg}
  if $\sansC(\alpha, \beta; \delta)$ holds in $\bA$, 
  then $\sansC(\alpha\cap B^2, \beta\cap B^2;\delta\cap B^2)$ holds in $\bB$;
\item \label{item:factors}
  if $\delta' \leq \delta$, then $\sansC(\alpha, \beta; \delta)$ holds 
  in $\bA$ iff $\sansC(\alpha/\delta', \beta/\delta'; \delta/\delta')$
  holds in $\bA/\delta'$.
\end{enumerate}
\end{lemma}
\begin{remark}
By (\ref{fact:centralizing_over_meet}), 
if $\alpha \meet \beta = 0_{\bA}$,  
then $\sansC(\beta, \alpha)$ and $\sansC(\alpha, \beta)$.
By (\ref{fact:dual_monotone_centralizers}),
if an algebra $\bA$ is abelian, 
then $\sansC(1_\bA, 1_\bA; \theta)$ for all $\theta \in \Con(\bA)$, so
in this case (\ref{item:factors}) implies that $\sansC(1_{\bA/\theta}, 1_{\bA/\theta})$ for every $\theta \in \Con(\bA)$.
\end{remark}

\ifthenelse{\boolean{long}}{%
\begin{lemma}
\label{lem:triv-clone-implies-abelian}
If $\Clo\bA$ is trivial (i.e., generated by the projections),
then $\bA$ is abelian.
\end{lemma}}{}
%% (An easy proof of Lemma~\ref{lem:triv-clone-implies-abelian} appears in 
%% Appendix Section~\ref{sec:proofs-elem-facts}.
%% It can also be shown that $\bA$ is \emph{strongly abelian} in this case,
%% but we won't need this.) 

\ifthenelse{\boolean{long}}{%
If the congruence lattice $\Con(\bA)$ has a 0,1-sublattice of shape
$M_n$ with $n\geq 3$,
then $\bA$ is abelian, as the next result shows.\footnote{A
``0,1-sublattice of shape
$M_n$ with $n\geq 3$'' is a height-two modular sublattice with top $1_\bA$, bottom
  $0_\bA$, and at least~3 atoms.}
For proof, see the Appendix.}{}
\begin{lemma}
\label{lem:M3-abelian}
If $\alpha_1$, $\alpha_2$, $\alpha_3 \in \Con(\bA)$ are pairwise complements,
then $\sansC(1_\bA, \alpha_i)$ for each $i=1,2,3$.  If, in addition, $\bA$ is
idempotent with a Taylor term operation, then $\sansC(1_\bA, 1_\bA)$; that is, $\bA$ is abelian.
\end{lemma}

\ifthenelse{\boolean{long}}{%
Here is another well known and easily derivable fact:
$\bA$ is abelian if and only if the diagonal of $\bA \times \bA$ is a class of a
congruence of $\bA \times \bA$.}{} 
We denote the diagonal of $A$ by $D(A) := \{(a,a): a \in A\}$. 
\begin{lemma}
\label{lem:diagonal}
An algebra $\bA$ is abelian if and only if there is some $\theta \in \Con (\bA^2)$ that has
the diagonal set $D(A)$ as a congruence class.
\end{lemma}

Lemma~\ref{lem:diagonal} can be used to prove that if there is 
a congruence of $\bA_1 \times \bA_2$ that has the graph of a bijection as a
block, then both $\bA_1$ and $\bA_2$ are abelian algebras.  This is the content of 
Lemma~\ref{lem:bijection_abelian}. 
\ifthenelse{\boolean{long}}{%
(See Appendix Section~\ref{sec:proofs-elem-facts} for proof.)}{
(See~\cite{Bergman-DeMeo} for proof.)}
\begin{lemma}
  \label{lem:bijection_abelian}
Suppose $\rho: A_1 \to A_2$ is a bijection and suppose the graph
$\{(x, \rho x) \mid x \in A_1\}$ is a block of some congruence
$\beta \in \Con (A_1 \times A_2)$.  Then both $\bA_1$ and $\bA_2$ are abelian.
\end{lemma}

\subsection{Prelude to tame congruence theory}
\label{sec:residuation-lemma}
Suppose $e \in \Pol_1\bA$ is a unary polynomial satisfying $e^2(x) = e(x)$ for
all $x\in A$.  Define $B=e(A)$ and
$F_B = \{ef\resB \suchthat f\in \Pol_1\bA\}$, and consider the
unary algebra $\bB= \<B, F_B\>$.
(In the definition of  $F_B$, we could have used
$\Pol \bA$ instead of $\Pol_1\bA$, and then our discussion would not be
limited to unary algebras.  However, we are mainly concerned with
congruence lattices, so we lose nothing by restricting the scope in this way.)

\Peter \Palfy and Pavel \Pudlak
prove in~\cite[Lemma~1]{Palfy:1980} that
the restriction mapping $\resB$, defined on $\Con\bA$ by
$\alpha\resB = \alpha \cap B^2$, is a lattice epimorphism of $\Con\bA$ onto $\Con\bB$.
In~\cite{McKenzie:1983}, Ralph McKenzie
developed the foundations of what would become tame congruence theory, and
the \PP\ lemma played a seminal role in this development.  In his presentation
of the lemma, McKenzie introduced the mapping $\hatmap$ defined on $\Con\bB$ as follows:
\begin{equation}
  \label{eq:hatmap}
\widehat{\beta} = \{(x,y) \in A^2 \suchthat \text{ for all }
f\in \Pol_1\bA, \, (ef(x), ef(y))\in \beta \}.
\end{equation}
Throughout this paper, whenever $\bA = \< A, \dots\>$
 and $\bB = \< B, \dots\>$ are algebras with $B = e(A)$ for some
$e^2 = e \in \Pol_1\bA$,  we take 
$\hatmap \colon \Con\bB \rightarrow \Con\bA$ to mean the map 
defined in~(\ref{eq:hatmap}).
It is not hard to see that the codomain of $\hatmap$ is indeed $\Con\bA$.  For
example, if $(x,y) \in \widehat{\beta}$ and $g\in \Pol_1\bA$, then for all 
$f\in \Pol_1\bA$ we have $(efg(x),efg(y)) \in \beta$, so 
$(g(x),g(y))\in \widehat{\beta}$.

For each $\beta \in \Con\bB$, let $\beta^* = \Cg^\bA(\beta)$.  That is,
$^*\colon \Con\bB \rightarrow \Con\bA$ is
the congruence generation operator restricted to the set $\Con\bB$.
The following lemma concerns the three mappings, $\resB$, $\hatmap$, and $^*$.
The third statement of the lemma, which follows from the first two,
will be useful in the later sections of the paper.

\begin{lemma}[Lemma 2.1 of \cite{overalgebras}]\
\label{lem:residuation}
\begin{enumerate}[\rm(i)]
  \item \label{item:residlemma-i} $^*\colon \Con\bB \rightarrow \Con\bA$ is a residuated mapping with
    residual $\resB$.
  \item \label{item:residlemma-ii} $\resB \colon  \Con\bA \rightarrow \Con\bB$ is a residuated mapping with
    residual $\hatmap$.
\item \label{item:residlemma-iii} For all $\alpha \in \Con\bA$, for all $\beta \in \Con\bB$,
\[
\beta = \alpha\resB \quad \Longleftrightarrow  \quad
\beta^* \leq \alpha \leq \widehat{\beta}.
\]
In particular,
$\beta^*\resB = \beta = \widehat{\beta}\resB$.
  \end{enumerate}
\end{lemma}
The proof is straightforward once we recall the definition of a residuated mapping.
However, as we will only make use of item (iii), we relegate the proof of 
Lemma~\ref{lem:residuation} to Section~\ref{sec:residuation-lemma-1} of the appendix.

The lemma above was inspired by the two approaches to
proving~\cite[Lemma~1]{Palfy:1980}.  In the original paper $^*$ is used, while
McKenzie uses the $\hatmap$ operator.  Both $\beta^*$ and
$\widehat{\beta}$ are mapped onto $\beta$ by the restriction map $\resB$, so
the restriction map is indeed onto $\Con\bB$.
By combining the two approaches, our version of the lemma highlights 
the fact that the interval 
\[
\lb\beta^*, \widehat{\beta}\rb =
\{\alpha \in \Con\bA \suchthat \beta^* \leq \alpha \leq \widehat{\beta}\}
\]
is precisely the set of congruences for
which $\alpha\resB = \beta$.  In other words, the
$\resB$-inverse image of $\beta$ is
$\lb\beta^*, \widehat{\beta}\rb$.
This fact plays a central role in the
theory developed in this  paper.
For the sake of completeness, we conclude this section by
verifying that~\cite[Lemma~1]{Palfy:1980} can be obtained from the lemma above.
\begin{corollary}%bd [cf.~Lemma 1 of \cite{Palfy:1980}]
[cf.~{\cite[Lemma 1]{Palfy:1980}}]
The mapping  $\resB \colon  \Con\bA \rightarrow \Con\bB$ is onto and preserves meets and joins.
\end{corollary}
\begin{proof}
  Given $\beta\in \Con\bB$, each $\theta\in \Con\bA$ in the interval 
  $\lb \beta^*, \widehat{\beta}\rb$ is mapped to $\theta\resB = \beta$, so $\resB$ is clearly
  onto.  That $\resB$ preserves meets is obvious, so we just check that $\resB$ is
  join preserving.  Since $\resB$ is order preserving, we have, for all 
  $S \subseteq \Con\bA$,
  \[
  \Join \theta\resB \leq \bigl(\Join \theta\bigr)\resB,
  \]
where joins are over all $\theta \in S$.  The opposite inequality follows
from~(\ref{eq:resid2}) above. Indeed, by~(\ref{eq:resid2}) we have
\[
\bigl(\Join \theta\bigr)\resB \leq \Join \theta\resB
\quad \Longleftrightarrow \quad
\Join \theta \leq \bigl(\Join \theta\resB \bigr)\!\widehat{\phantom{X}}
%%wjd: yes, this is fine (and I'll change it below as well).
\]
and the last inequality holds by another application of~(\ref{eq:resid2}): if $\eta \in S$, then
\[
\eta \leq \bigl(\Join \theta\resB\bigr)\!\widehat{\phantom{X}}
\quad \Longleftrightarrow \quad
\eta\resB \leq \Join \theta\resB.
\]
\end{proof}

\section{Tame Congruence Theory}
In this section we review some basic definitions and results of \emph{tame congruence theory},
as developed by David Hobby and Ralph McKenzie in \cite{HM:1988}. 
We then remark on some consequences of this theory that is useful for our application.
In particular, we will see that if $\bA$ is a finite algebra whose
congruence lattice is $M_n$, for some $n>2$, then $\bA$ is Abelian. 
(Note how this improves upon~\ref{lem:M3-abelian}.)
Moreover, we will see that when $n-1$ is not a prime power, 
$\Con \bA \cong M_n$ implies that $\bA$ is \emph{strongly Abelian}.

\subsection{Tight lattices} In this section we define the class of ``tight'' lattices
(\cite[Definition 1.6]{HM:1988}) and the class of ``tame'' congruences. 
Later we apply these concepts to congruence lattices of shape $M_n$.

\begin{definition}
Let $L$ be any lattice.
\begin{enumerate}
\item 
By a \defn{meet endomorphism} of $L$ we mean a function 
$\mu : L \to L$ satisfying, for all $x, y \in L$,
$\mu(x\meet y) = \mu(x) \meet \mu(y)$.
A \defn{join endomorphism} is defined dually.
\item Departing from standard terminology, 
we call a function $\mu: L \to L$ \defn{increasing}
iff $\mu(x) \geq x$ for all $x in L$, and 
\defn{strictly increasing}
if $\mu(x) > x$ for all $x\in L$ except the largest element of $L$.
The concepts of \defn{decreasing} and of \defn{strictly decreasing}
function on $L$ are defined dually.
\item By a \defn{polarity} of $L$ we mean a pair $(\sigma, \mu)$ 
such that $\sigma$ is a decreasing join endomorphism of $L$, $\mu$ 
is an increasing meet endomorphism of $L$, and
$\sigma\mu(x) \leq x \leq \mu\sigma(x)$ for all $x$ in $L$.
%% \item By a \defn{tolerance} of $L$, we mean a reflexive and symmetric subalgebra of 
%% $L^2$, i.e., a binary relation p С L2 such that for all x, j/, w, v ? L we have : (i) (x, x) ? p\
%% (ii) (x,y) € p iff (y,x) ? p\ (iii) if (x,y), (w, v) ? p then (xW u^yW v) ? p and
%% (x Ли,у Av) ? p.
\end{enumerate}
\end{definition}
%% In finite lattices, there are one-to-one correspondences: 
%% tolerances $\leftrightarrow$ polarities 
%% $\leftrightarrow$ increasing meet endomorphisms 
%% $\leftrightarrow$ decreasing join endomorphisms.
%% (See \cite[Lemma 1.2]{HM:1988}.)

%% \begin{lemma}[Lemma 1.3 of \cite{HM:1988}]
%% Let $\rho$ be a tolerance of a finite lattice $L$, and let $(\sigma, \mu)$ 
%% be the associated polarity. The following are equivalent.
%% A) $\rho$ is connected.
%% B) There exists a sequence 0 = xo < x\ < • - • < xn = 1 of elements of L (for
%% some n > 0) with (xi,Xi+i) ? $\rho$ for all i < n.
%% C) cr is strictly decreasing.
%% D) /x is strictly increasing.
%% \end{lemma}

Let $L$ be any lattice with $0$ and $1$. 
A homomorphism $f: L\to L'$
is called 0,1-\defn{separating} iff $f^{-1}\{f(0)\} = \{0\}$ ($f$ \emph{separates} 0) and 
$f^{-1}\{f(1)\}  = \{1\}$ ($f$ \emph{separates} 1). 
%% To denote that $f$ is a homomorphism with the property just defined,we write / : L '—> I/. 
We say that $L$ is 0,1-\defn{simple} iff $|L| > 1$ and every 
nonconstant homomorphism $f : L \to L'$ is 0,1-separating.

\begin{definition}
A lattice $L$ is called \defn{tight} iff $L$ is finite, 
$|L| > 1$, and if $\rho$ is any tolerance of $L$ containing $\<0,a\>$ for some $a > 0$, 
or containing  $\<b,1\>$ for some $b < 1$, then $\rho= L^2$.
\end{definition}

\begin{lemma}[Lemma 1.7 of \cite{HM:1988}]
\label{lem:tight}
A finite lattice $L$ is tight iff $L$ is 0,1-simple and every strictly 
increasing meet endomorphism of $L$ is constant (i.e., $L^2$ is the only 
connected tolerance of $L$).
\end{lemma}

\begin{lemma}[Lemma 1.8 of \cite{HM:1988}]
For any lattice $L$ with $0$ and $1$, such that $|L| > 1$, the following are
equivalent.
\begin{enumerate}
\item $L$ is 0,1-simple.
\item $L$ has a largest congruence $\theta \neq L^2$, and this congruence satisfies 
$1/\theta = \{1\}$ and $0/\theta = \{0\}$.
\end{enumerate}
\end{lemma}
From Lemma~\ref{lem:tight}, it easily follows that for every $n \geq 2$, the lattice 
$M_n$ is tight. 

%% \begin{definition}[Definition 2.2 of \cite{HM:1988}]
%% Suppose that $A$ is a nonempty set, $\emptyset \neq U \subseteq A$, 
%% $\theta \in \Eq(A)$, $f$ is a function with domain $A$,
%% $h:A^n\to A$ is an $n$-ary operation on $A$, and 
%% $U_0 \cup \cdots \cup U_{n-1} \subseteq A$, and 
%% denote by $\underline{U}$ (resp., $U^n$) the usual Cartesian product $U_0 \times \cdots U_{n-1}$
%% (resp., power $U \times \cdots \times U$).
%% Define
%% \begin{enumerate}
%% \item $\theta\resU := \theta \cap U^2$;
%% \item $f|_U := \{(x,f(x)) \colon x \in U\}$;
%% \item $h|_{\underline{U}} := \{(x_0,\dots, x_{n-1}, h(x_0,\dots, x_{n-1})) \colon x_i \in U_i, 0 \leq i<n\}$;
%% \item $h|_U:= h|_{U^n}$.

%% If $\bA = \<A,\dots\>$ is any algebra with universe $A$, define
%% \item $(\Pol \bA)|_U := $ the set of all $h|_U$ such that $h\in \Pol_n\bA$ 
%% for some $n$, and $h(U^n) \subseteq U$;
%% \item $\bA|_U := \<U, (\Pol\bA)|_U\>$, called the \defn{algebra induced on $U$ by $\bA$} 
%% (or an \defn{induced algebra of $\bA$}).
%% \end{enumerate}
%% \end{definition}



\subsection{Tame quotients}
All algebras considered will be assumed to be finite. 
The concept of a minimal set relative to a pair of congruences is fundamental.

\begin{definition}[Definition 2.5 of \cite{HM:1988}]
Let $\bA$ be a finite algebra and let $\alpha < \beta$ be two congruences of $\bA$. 
Define $U_\bA (\alpha, \beta)$ to be the set of all sets of the form 
$f(A)$ where $f\in \Pol_1\bA$ and $f(\beta)\nsubseteq \alpha$.
Define $M_\bA (\alpha, \beta)$ to be the set 
of all minimal members of $U_\bA (\alpha, \beta)$.
Call the members of $M_\bA (\alpha, \beta)$ the 
$\alpha,\beta$-\defn{minimal sets} of $\bA$.
\end{definition}
\newcommand\Mab{\ensuremath{M_\bA (\alpha, \beta)}}
Observe that $\Mab$ is non-empty and for each $\alpha,\beta$-minimal set 
$U$, we have $\alpha|_U \neq \beta|_U$.
By a \defn{quotient} in a lattice $L$, we mean simply a pair $\<x,y\>$ 
of elements of $L$ with $x < y$. 
We often refer to such a pair as ``the quotient $\q{x}{y}$ in $L$.'' 
A \defn{prime quotient} is a covering relation $x\prec y$ in $L$; that is,  
if $x\leq z\leq y$, then either $z=x$ or $z=y$.
The interval lattice $\lb x,y \rb$ associated with a quotient $\q{x}{y}$ 
is the sublattice of $L$ consisting of all elements $z$ such that 
$x \leq z \leq y$. 

\begin{definition}
For a finite algebra $\bA$, a quotient $\qab$
in $\Con \bA$ is called \defn{tame} if there exists $V \in \Mab$ 
and a unary $e^2=e \in \Pol_1\bA$ such that $e(A) = V$ and 
such that $\lb \alpha, \beta \rb \to \lb \alpha\resV, \beta\resV \rb$ 
is a 0,1-separating lattice homomorphism.
\end{definition}

\begin{theorem}[Thm.~2.11 of \cite{HM:1988}]
If $\qab$ is a quotient in $\Con \bA$ such that $\lb \alpha, \beta \rb$ is tight,
then $\qab$ is tame.
\end{theorem}

\newcommand{\abmin}{$\alpha, \beta$-minimal\xspace}

Let $\qab$ be a quotient in $\Con\bA$.
If $A \in \Mab$, then we call $\bA$ \defn{minimal relative to} 
$\qab$, or simply $\alpha, \beta$-\defn{minimal}.
\begin{lemma}[Lemma 2.13 of \cite{HM:1988}]
If $\qab$ is a quotient in $\Con\bA$, then the following hold.
\begin{enumerate}[(1)]
\item $\bA$ is \abmin iff for all $f\in \Pol_1\bA$, 
  either $f$ is a permutation of $\bA$ or $f(\alpha) \subseteq \beta$.
\item If $\bA$ is \abmin then $\qab$ is tame.
\item \label{item:1}
  If $\qab$ is tame and $U \in \Mab$ then the algebra 
  $\bA|_U$ is $\alpha|_U, \beta|_U$-minimal.
\end{enumerate}
In item (\ref{item:1}), ``tameness'' can be replaced by ``there exists 
$e^2 = e\in \Pol_1\bA$ with $U = e(A)$.''
\end{lemma}

A finite algebra $\bA$ is called \defn{minimal} if 
$\bA$ is $0_A, 1_A$-minimal, equivalently, $|A|>1$ and every nonconstant 
$f\in \Pol_1\bA$ is a permutation. 
A finite algebra $\bA$ is called \defn{E-minimal} if 
$|A|> 1$ and every nonconstant $e^2 = e\in \Pol_1\bA$ is
the identity.

\subsection{The types of tame quotients}
We are now ready to define and study the five types of tame congruence quotients.
In this section we delineate the distinct characters of these types, primarily in
relation to the ``polynomial structure'' of an algebra. Later, we
consider the congruence lattice of an algebra as a labeled graph, where all of
the prime quotients are labeled with their respective types. We then consider
the ways in which this labeling is influenced by the unlabeled congruence lattice,
construed purely as an abstract lattice.

\begin{definition}[Definition 5.1 of \cite{HM:1988}]\
  \begin{enumerate}
\item Let $\qab$ be a tame quotient of congruences in a finite algebra 
$\bA$. Let $U\in \Mab$.  We define the \defn{type} of $\qab$, 
written $\typ(\alpha,\beta)$, to be the type of $\bA\resU$ 
relative to $\q{\alpha\resU}{\beta\resU}$.
\item
Let $\q{\gamma}{\lambda}$ be any quotient of congruences in a finite algebra $\bA$. 
Then we let $\typ\{\gamma, \lambda\}$ denote the set
$\{\typ(\alpha, \beta) : \gamma \leq \alpha \prec \beta \leq \lambda\}$.
\item
Let $\bA$ be any finite algebra. We call $\bA$ \defn{tame} iff the quotient 
$\q{0_A}{1_A}$ is tame. % (implying that $|A| > 1$). 
If $\bA$ is tame, we put $\typ \bA = \typ(0_A, 1_A)$.
\item Let $\bA$ be any finite algebra. By $\typ\{\bA\}$ we denote the set 
$\typ\{0_A,1_A\}$ of types.
  \end{enumerate}
\end{definition}

%% The type of a tame quotient $\qab$ in a finite algebra $\bA$ is well-defined by the
%% above. To verify this, let Uo and U\ be (a,/?)-minimal sets. According to Theo-
%% Theorem 2.8 A) and Exercise 2.9 E), there exists an isomorphism between the structures
%% (Uo,Po\A\Uo,a\Uo,C\u0) and (C/bPolAl^al^,/?!^). Therefore the type of A|^
%% relative to {pt\un0\ui) is the same for г = 0 and г = 1.
%% A first corollary of this definition and of our earlier work is worth noting. Recall
%% that for a tame quotient (a,/3) in A, an (a,/?)-trace is simply any set N such that
%% for some U G MA(a, /?) and x G C/, we have N = (x//3) r\U Ф (х/а) П СЛ
%% COROLLARY 5.2. Let (a,/?) be a tame quotient in a unite algebra A.
%% A) For every (a,/3)-trace N, the algebra M = (A\n)/(cx\n) & a minimal (and
%% therefore tame) algebra, and typ(a,/3) = typ(M).
%% B) If typ(a, /3) Ф 1 or if а -< /3, then for every pair of (a, ft)-traces No and N\,
%% we have No ^ N1 and Mo = Mi, where M{ =

%% The Abelian types are 1 and 2; the non-Abelian types are 3, 4 and 5. The next
%% theorem summarizes what we have learned thus far about the types, and contains a
%% new result that has some interesting corollaries. After looking at these corollaries, we
%% shall begin a study of the non-Abelian types. Recall that a quotient (a, ft) is called
%% prime iff a -< /?, i.e., /? covers a.

\begin{theorem}[Theorem 5.7 of \cite{HM:1988}] 
\label{thm:types-tame-quotients}
Let $\bA$ be a finite algebra.  
\begin{enumerate}[\rm(1)]
\item Every prime congruence quotient of $\bA$ is tame.
\item \label{item:2}
  For any quotient $\qab$ of $\bA$, the following are equivalent:
  \begin{enumerate}[\rm(i)]
  \item $\qab$ is prime and nonabelian.
  \item $\qab$ is tame and $\typ(\alpha, \beta)\in \{3,4,5\}$.
  \end{enumerate}
\item A tame quotient $\qab$ has type 1 iff it is strongly Abelian, and has type 2 
  iff it is Abelian but not strongly Abelian.
\item  \label{item:3}
  For any quotient $\qab$ of $\bA$ that is not strongly Abelian, the following are
  equivalent:
  \begin{enumerate}[\rm(i)]
  \item $\qab$ is tame.
  \item The interval lattice $\lb\alpha, \beta \rb$ is tight.
  \item $\lb\alpha, \beta \rb$  is 0,1-simple and complemented.
  \item  $\lb\alpha, \beta \rb$  admits a 0,1-separating homomorphism onto the congruence 
    lattice of a vector space. (This homomorphism is essentially unique.)
  \end{enumerate}
\end{enumerate}
\end{theorem}

\subsection{Application to $M_n$}
%% Let $\bA$ be a finite algebra with congruence lattice $\Con \bA \cong M_n$.
We first note that, for $n\geq 2$, the lattice $M_n$ 
is simple and tight.
%% \begin{lemma}The lattice $M_n$ is tight.\end{lemma}
Suppose $\bA$ is a finite algebra with $\Con \bA\cong M_n$. 
Then $\lb 0_A,1_A \rb$ is tight, hence $\q{0_A}{1_A}$ is tame. 
Since $\q{0_A}{1_A}$ is not prime, by 
Theorem~\ref{thm:types-tame-quotients}(\ref{item:2}),
$\typ(0,1) \notin \{3,4,5\}$. Thus, $\bA$ is Abelian.
Finally, if $\bA$ is Abelian, but not strongly abelian, then by 
Theorem~\ref{thm:types-tame-quotients}(\ref{item:3}),
$M_n$ must be isomorphic to the congruence lattice of a vector space, 
hence $n-1$ is a prime power.

\section{Graphical Compositions}
Let $X$ be a nonempty set. Given $\alpha$, $\beta$, $\gamma \in \Eq(X)$,
define the relation 
%% $R(\alpha, \beta, \gamma)\subseteq X\times X$ as
%% follows:
\[
W(\alpha, \beta; \gamma) := \{(x,y) \in X^2 : \exists (a, b) \in \gamma
\; . \; x \ralpha a \rbeta y \ralpha b \rbeta x\}.
\]
That is, $(x,y) \in W(\alpha, \beta; \gamma)$ iff there exists $(a,b)\in \gamma$
such that the relations in Figure~\ref{fig:wsb} hold.


%\begin{center}
\begin{figure}[!h]
      \begin{tikzpicture}[scale=1.2]
      \node (x) at (-1,1)  [draw, circle, inner sep=\dotsize] {};
      \node (y) at (1,1)  [draw, circle, inner sep=\dotsize] {};
      \node (a) at (0,2)  [draw, circle, inner sep=\dotsize] {};
      \node (b) at (0,0)  [draw, circle, inner sep=\dotsize] {};
      \draw (x) node [left] {$x$};
      \draw (a) node [above] {$a$};
      \draw (b) node [below] {$b$};
      \draw (y) node [right] {$y$};
      \draw[semithick] (b)-- (a) node[pos=.5,right] {$\gamma$};
      \draw[semithick] (x)-- (a) node[pos=.5,left] {$\alpha$};
      \draw[semithick] (x)-- (b) node[pos=.5,left] {$\beta$};
      \draw[semithick] (b)-- (y) node[pos=.5,right] {$\alpha$};
      \draw[semithick] (a)-- (y) node[pos=.5,right] {$\beta$};
      \end{tikzpicture}
  \caption{The Wheatstone Bridge defines the relation 
    $W(\alpha, \beta; \gamma)$. }
  \label{fig:wsb}
\end{figure}
%\end{center}
It is not hard to see that $W(\alpha, \beta; \gamma)$ is a 
\emph{tolerance} (that is, a reflexive symmetric relation)
on $X$.  Also obvious is the fact that 
$W(\alpha, \beta; \gamma) = W(\beta, \alpha; \gamma)$.

Suppose $\alpha_1$, $\alpha_2$, and $\alpha_3$ are \pppc in $\Eq(X)$, and
let $\beta\in \Eq(X)$ be complementary to 
each 
$\alpha_i$, so that $L = \{0_X, \alpha_1, \alpha_2, \alpha_3, \beta, 1_X\} \cong M_4$. (See Figure \ref{fig:M4}.)


\begin{figure}[!h]
      \begin{tikzpicture}[scale=1.2]
      \node (bot) at (0,0)  [draw, circle, inner sep=\dotsize] {};
      \node (a1) at (-1.25,1)  [draw, circle, inner sep=\dotsize] {};
      \node (a2) at (-.5,1)  [draw, circle, inner sep=\dotsize] {};
      \node (a3) at (.5,1)  [draw, circle, inner sep=\dotsize] {};
      \node (b) at (1.25,1)  [draw, circle, inner sep=\dotsize] {};
      \node (top) at (0,2)  [draw, circle, inner sep=\dotsize] {};
      \draw (a1) node [left] {$\alpha_1$};
      \draw (a2) node [left] {$\alpha_2$};
      \draw (a3) node [left] {$\alpha_3$};
      \draw (b) node [right] {$\beta$};
      \draw (bot) node [below] {$0_X$};
      \draw (top) node [above] {$1_X$};
      \draw[semithick] (bot)--(a1)--(top)--(a2)--(bot)--(a3)--(top)--(b)--(bot);
      \end{tikzpicture}
  \caption{The lattice $M_4$.}
  \label{fig:M4}
\end{figure}
Define the relation 
$\tau=\tau(\alpha_1, \alpha_2, \beta)\subseteq X\times X$ as
$R(\alpha_1, \alpha_2, \beta)\subseteq X\times X$ as
follows:
\[
x \rtau y \quad \longleftrightarrow \quad (\exists (a, b) \in \beta)\;  
x \ralpha_1 a \ralpha_2 y \ralpha_1 b \ralpha_2 x.
\]
Graphically, $x \mathrel{\tau} y$ if and only if there exist $a, b \in X$
satisfying the relations depicted in Figure~\ref{fig:rho}.
%\begin{center}
\begin{figure}
      \begin{tikzpicture}[scale=1.2]
      \node (x) at (-1,1)  [draw, circle, inner sep=\dotsize] {};
      \node (y) at (1,1)  [draw, circle, inner sep=\dotsize] {};
      \node (a) at (0,2)  [draw, circle, inner sep=\dotsize] {};
      \node (b) at (0,0)  [draw, circle, inner sep=\dotsize] {};
      \draw (x) node [left] {$x$};
      \draw (a) node [above] {$a$};
      \draw (b) node [below] {$b$};
      \draw (y) node [right] {$y$};
      \draw[semithick] (b)-- (a) node[pos=.5,right] {$\beta$};
      \draw[semithick] (x)-- (a) node[pos=.5,left] {$\alpha_1$};
      \draw[semithick] (x)-- (b) node[pos=.5,left] {$\alpha_2$};
      \draw[semithick] (b)-- (y) node[pos=.5,right] {$\alpha_1$};
      \draw[semithick] (a)-- (y) node[pos=.5,right] {$\alpha_2$};
      \end{tikzpicture}
  \caption{The Wheatstone Bridge which defines the relation 
    $\tau(\alpha_1, \alpha_2, \beta)$ as follows: 
    $(x,y) \in \tau(\alpha_1, \alpha_2, \beta)$ if and only if there 
    exist $a, b \in X$ satisfying the relations in the diagram.}
  \label{fig:rho}
\end{figure}
%\end{center}
It is clear that $\tau$ is a \emph{tolerance}, that is, a reflexive and symmetric binary relation.
Let $f: X\to X$ be a unary operation and suppose that $f$ is \emph{compatible} with each
relation $\theta \in \{\alpha_1, \alpha_2, \beta\}$, that is, 
$(u,v)\in \theta \, \longrightarrow \, (f(u), f(v))\in \theta$.  Then $f$ is also
compatible with $\tau$. %% (Consider the diagram in Figure~\ref{fig:rho}, and give
%% each vertex $u$ the label $f(u)$.)

\begin{fact} If 
$L = \{0_X, \alpha_1, \alpha_2, \alpha_3, \beta, 1_X\} \cong  M_4$,
then
  \[
\alpha_1 \cap \tau(\alpha_1, \alpha_2, \beta)
= 0_X =  \alpha_2 \cap \tau(\alpha_1, \alpha_2, \beta),
\]
\[
\alpha_1 \cap \tau(\alpha_1, \alpha_3, \beta)
= 0_X =  \alpha_3 \cap \tau(\alpha_1, \alpha_3, \beta),
\]
\[
\alpha_2 \cap \tau(\alpha_2, \alpha_3, \beta)
 = 0_X = \alpha_3 \cap \tau(\alpha_2, \alpha_3, \beta).
\]
%%   \[
%% \tau(\alpha_1, \alpha_2, \beta) \cap \alpha_1
%% = 0_X = \tau(\alpha_1, \alpha_2, \beta) \cap \alpha_2,
%% \]
%% \[
%% \tau(\alpha_1, \alpha_3, \beta) \cap \alpha_1
%% = 0_X = \tau(\alpha_1, \alpha_3, \beta) \cap \alpha_3,
%% \]
%% \[
%% \tau(\alpha_2, \alpha_3, \beta) \cap \alpha_2
%%  = 0_X = \tau(\alpha_2, \alpha_3, \beta) \cap \alpha_3.
%% \]
\end{fact}
\begin{proof}
  Fix $(x,y) \in  \alpha_1 \cap \tau(\alpha_1, \alpha_2, \beta)$ and suppose 
  $a$ and $b$ satisfy the diagram in Figure~\ref{fig:rho}.  Then 
  $(x,y) \in \alpha_1$ implies $(a, b)\in \alpha_1 \meet \beta = 0_X$, so 
  $a = b$.  Therefore, $(x,y) \in \alpha_1 \meet \alpha_2 = 0_X$, so $x = y$.
  Proofs of the other identities are similar.
\end{proof}

\section{Functions Derived from Graphical Compositions}
Let $R_{12}^\beta$ be the relation on $X^2 \times X^2$ defined by 
\[
(a,b) \rel{R_{12}^\beta} (x,y) \quad \longleftrightarrow \quad 
(a,b) \in \beta \; \text{ and } \;
x \ralpha_1 a \ralpha_2 y \ralpha_1 b \ralpha_2 x.
\]
Define $R_{13}^\beta$ and $R_{23}^\beta$ similarly.  Graphically, 
$(a,b) \rel{R_{12}^\beta} (x,y)$ holds if and only if the relations in
Figure~\ref{fig:rho} are satisfied.
\begin{lemma}
\label{lem:injection}
Suppose $\alpha_i$ and $\alpha_j$ are complementary equivalence
relations on $X$ with uniform blocks of size $\sqrt{|X|}$.
Then the relation $R_{ij}^\beta$ restricted to $\beta\times X^2$ is
  a one-to-one function from $\beta$ into $X^2$.
%$R_{ij}^\beta : \beta   \rightarrow X^2$.
\end{lemma}
\begin{proof}
First we note that each pair $(a,b)\in \beta$ has at most one image. For if
$(a,b) \rel{R_{ij}^\beta} (x,y)$ and $(a,b) \rel{R_{ij}^\beta} (u,v)$, then 
$(x,u) \in  \alpha_i \meet \alpha_j = 0_X$ and 
$(y,v) \in  \alpha_i \meet \alpha_j = 0_X$, so $(x,y) = (u,v)$.

Next,
since both $\alpha_i$ and $\alpha_j$ have 
$\sqrt{|X|}$ blocks, and since each of these blocks has size $\sqrt{|X|}$, we
see that each block of $\alpha_i$ intersects each block of $\alpha_j$ at
exactly one point.  That is, for all $a, b \in X$, the set 
$a/\alpha_i \cap b/\alpha_j$ is a singleton.
Therefore, to each $(a,b)\in \beta$ there corresponds precisely one $(x,y)\in
X^2$ such that $(a,b) \rel{R_{ij}^\beta} (x,y)$ holds.  
Specifically, $\{x\} = a/\alpha_i \cap b/\alpha_j$ and 
$\{y\} = b/\alpha_i \cap a/\alpha_j$. 
Thus,  $R_{ij}^\beta$ is a function.  

From now on, we let 
$R_{ij}^\beta((a,b))$ denote the image of $(a,b)$ under $R_{ij}^\beta$; that
is, $R_{ij}^\beta((a,b))$ denotes the ordered pair $(x,y)$
satisfying $(a,b) \rel{R_{ij}^\beta} (x,y)$.


Suppose $R_{ij}^\beta((a,b)) = R_{ij}^\beta((c,d))$. Then 
$(a,c) \in \alpha_i\meet \alpha_j = 0_X$
and
$(b,d) \in \alpha_i\meet \alpha_j = 0_X$, so $(a,b) = (c,d)$.
Therefore,  $R_{ij}^\beta$ is one-to-one.
\end{proof}

If, in addition to the assumptions  of Lemma~\ref{lem:injection}, we assume that
the image of $\beta$ under $R_{ij}^\beta$ is contained in $\beta$, then 
$R_{ij}^\beta: \beta \rightarrow \beta$ is a bijective involution.
That is, $R_{ij}^\beta$ is one-to-one and onto, and 
$R_{ij}^\beta\circ R_{ij}^\beta$ is the identity map.
%% \begin{lemma}
%% Let $\{\alpha_i : 0\leq i < r\}$ be a set of pairwise complementary
%% equivalence relations on $X$.
%% \begin{enumerate}
%% \item If $\alpha_1 \circ \alpha_2 = \alpha_2 \circ \alpha_1$, then 
%% $\alpha_1$ and $\alpha_2$ have uniform blocks.
%% \item If $\alpha_1$ and $\alpha_2$ have uniform blocks of size $|X|^{1/2}$, then 
%% $\alpha_1 \circ \alpha_2 =  \alpha_2 \circ \alpha_1$.
%% \item Three pairwise complementary equivalence relations are pairwise permuting if and only
%%   if all three have uniform blocks of size $|X|^{1/2}$.
%% \end{enumerate}
%% these three
%% relations have complementary structure, and each
%% $\alpha_i$ has block size $|X|^{1/2} = n$, for some positive integer
%% $n$.  Thus, the number of blocks of each $\alpha_i$ is $|X|^{1/2}$, and
%% $|X| = n^2$.
%% \end{lemma}

%% To answer the Palfy-Saxl question affirmatively, it seems it would be enough to
%% show that if $L = \{0_X, \alpha_1, \dots, \alpha_{n-1}, \beta, 1_X\} \cong M_n$
%% is a congruence lattice and if $\alpha_1$, $\alpha_2$, and $\alpha_3$ are \PPPC, and if 
%%  $R_{ij}^\beta: \beta \rightarrow \beta$ for each $i\neq j$ in $\{1, 2, 3\}$,
%% then the congruence relation $\beta$ contains exactly $|\beta| = |X|^{3/2}$
%% ordered pairs, and thus has the same block structure as, and permutes with, $\alpha_i$ for $i\in
%% \{1,2,3\}$.

\section{Final piece of the puzzle}
As above, suppose $L = \{0_X, \alpha_1, \alpha_2, \alpha_3, \beta, 1_X\} \cong  M_4$ is a
congruence lattice and suppose $\{\alpha_i\}_{i=1}^3$  is \pppc.
Suppose $R_{ij}^\beta: \beta \rightarrow \beta$ holds for all 
$i, j \in \{1,2,3\}$.

\begin{lemma}
  \label{lem:missingpiece}
If %$(a,w) \in \alpha_1\circ \beta$ with 
$a \ralpha_1 z \rbeta w$,
then one of the following holds:
\begin{enumerate}
\item $(a,w) \in \alpha_2$, 
\item $(a,w) \in \alpha_3$, 
\item $(a,w) \in \beta$, 
\item $a/\alpha_2 \cap z/\alpha_3 \cap w/\alpha_1 \neq \emptyset$,
\item $a/\alpha_3 \cap z/\alpha_2 \cap w/\alpha_1 \neq \emptyset$.
\end{enumerate}
\end{lemma}

If Lemma~\ref{lem:missingpiece} is true, then we can prove the following:
\begin{theorem}
If $L = \{0_X, \alpha_1, \alpha_2, \alpha_3, \beta, 1_X\} \cong  M_4$ is a
congruence lattice with $\alpha_i$ \pppc, then $\beta$ permutes with each $\alpha_i$.
\end{theorem}
\begin{proof}
We will show $\alpha_1 \circ \beta \subseteq \beta \circ \alpha_1$.
Assume $a \ralpha_1 z \rbeta w$.  We consider each of the cases in
Lemma~\ref{lem:missingpiece} in turn and,
in each case, find $b$ satisfying $a \rbeta b \ralpha_1 w$.
\begin{enumerate}
\item If $(a,w) \in \alpha_2$, then let $b = z/\alpha_2 \cap w/\alpha_1$.  Then
$R^\beta_{12}(z,w) = (a,b)$ and since 
$R^\beta_{12}: \beta \rightarrow \beta$, we have $(a,b) \in \beta$, so 
$a \rbeta b \ralpha_1 w$, as desired.  
\item If $(a,w) \in \alpha_3$, then let $b = z/\alpha_3 \cap w/\alpha_1$. Use the same
argument as in the first case, but replace $R^\beta_{12}$ with $R^\beta_{13}$. 
\item If $(a,w) \in \beta$, then let $b = a$. 
\item If  $a/\alpha_2 \cap z/\alpha_3 \cap w/\alpha_1 \neq \emptyset$, then let
  $y$ denote the element in this set.  Let $x = z/\alpha_1 \cap w/\alpha_3$, and
  let $b = x/\alpha_2\cap y/\alpha_1$.  
  Then $(R^\beta_{12}\circ R^\beta_{13})(z,w) = R^\beta_{12}(x,y) = (a,b)$, so $(a,b) \in \beta$.
  Now, $b\ralpha_1 y \ralpha_1 w$, so
  $a \rbeta b \ralpha_1 w$, as desired.
\item If $a/\alpha_3 \cap z/\alpha_2 \cap w/\alpha_1 \neq \emptyset$, then let 
  $y$ denote this element, let $x = z/\alpha_1 \cap w/\alpha_2$, and
  let $b = x/\alpha_3\cap y/\alpha_1$.  
  Then $(R^\beta_{13}\circ R^\beta_{12})(z,w) = R^\beta_{12}(x,y) = (a,b)$, so $(a,b) \in \beta$.
  Now, $b\ralpha_1 y \ralpha_1 w$, so $a \rbeta b \ralpha_1 w$, as desired.
\end{enumerate}
\end{proof}

\section{Proof of Lemma 3}
Consider the relation $\theta_{ij}$ defined as follows:
\[
x \rtheta_{ij} y \quad \longleftrightarrow \quad (\exists a, b) \;
a \ralpha_i x \ralpha_j b \rbeta y \ralpha_j a.
\]
Easy arguments similar to those above establish that 
\[
\theta_{ij} \cap \alpha_i = \theta_{ij} \cap \alpha_j =
\theta_{ij} \cap \beta = 0_X.\]
On the other hand, since $L$ is a congruence lattice, it must be the case that
the transtive closure of $\theta_{ij}$ is contained in $L$.\\
\\
{\it TODO: complete proof of Lemma 3 (if possible).}

\appendix
\acresetall

\section{Miscellaneous Proofs}
\label{sec:proofs-elem-facts}
\begin{lem} {\bf \ref{lem:triv-clone-implies-abelian}.}
If $\Clo \bA$ is trivial (i.e., generated by the projections),
then $\bA$ is abelian.
\end{lem}
\begin{proof}
We want to show $\sansC(1_\bA, 1_\bA)$.  Equivalently, we must show
that for all $t\in \Clo\bA$ (say, $(\ell+m)$-ary) 
and all $a, b \in A^\ell$, we have $\ker t(a,\cdot)=\ker t(b,\cdot)$.
We prove this by induction on the height of the term $t$.  Height-one terms are
projections and the result is obvious for these.  Let $n>1$ and assume the result
holds for all terms  of height less than
$n$.  Let $t$ be a term of height $n$, say, $k$-ary.  Then for some terms 
$g_1, \dots, q_k$ of height less than $n$ and for some $j\leq k$, we have
$t = p^k_j [q_1, q_2, \dots, q_k] = q_j$ and since $q_j$ has height less than
$n$, we have
\[
\ker t(a,\cdot)=\ker g_j(a,\cdot) = \ker g_j(b,\cdot)=\ker t(b,\cdot).
\]\end{proof}
In fact, it can be shown that $\bA$ is \emph{strongly abelian} in this case. 

\begin{lem}{\bf \ref{lem:M3-abelian}.}
If $\alpha_1$, $\alpha_2$, $\alpha_3 \in \Con(\bA)$ are pairwise complements,
then $\sansC(1_\bA, \alpha_i)$ for each $i=1,2,3$.  If, in addition, $\bA$ is
idempotent and has a Taylor term operation, then $\sansC(1_\bA, 1_\bA)$; that is, $\bA$ is abelian.
\end{lem}
\begin{proof}
  The goal is to prove $\sansC(1_\bA, 1_\bA)$.
  By Lemma~\ref{lem:centralizers}~(\ref{fact:centralizing_over_meet}), we have
  $\sansC(\alpha_1, \alpha_2; \alpha_1 \meet \alpha_2)$.  
  Since $\alpha_1 \meet \alpha_2= 0_\bA$, this means
  $\sansC(\alpha_1, \alpha_2)$.
  Similarly, $\sansC(\alpha_3, \alpha_2)$.  Therefore, by 
  Lemma~\ref{lem:centralizers}~(\ref{fact:centralizing_over_join1}), we have
  $\sansC(\alpha_1 \join \alpha_3, \alpha_2)$. This is equivalent to 
  $\sansC(1_\bA, \alpha_2)$, since $\alpha_1 \join \alpha_3 = 1_\bA$. 
  The same argument \emph{mutatis-mutandis} yields
  $\sansC(1_\bA,\alpha_1)$ and $\sansC(1_\bA,\alpha_3)$. 
  Before proceding, note that $\sansC(\alpha_1, \alpha_1)$, by 
  Lemma~\ref{lem:centralizers}~(\ref{fact:monotone_centralizers1}).
  Now, if $\bA$ is idempotent and has a Taylor term operation, then
  by \ref{thm:kearnes-kiss-3.27} we have 
  $\sansC(\alpha_1 \join \alpha_2,\alpha_1 \join \alpha_2; \alpha_2)$.
  That is, $\sansC(1_\bA,1_\bA; \alpha_2)$.
  Similarly, $\sansC(1_\bA,1_\bA; \alpha_3)$.
  By~\ref{lem:centralizers}~(\ref{fact:centralizing_over_meet2}) then, 
  $\sansC(1_\bA,1_\bA; \alpha_2 \meet\alpha_3)$. 
  That is, $\sansC(1_\bA,1_\bA)$.
\end{proof}

\begin{lem}{\bf\ref{lem:diagonal}.}
 An algebra $\bA$ is abelian if and only if there is some 
 $\theta \in \Con (\bA^2)$ that has the diagonal $D(A):= \{(a,a): a \in A\}$ 
 as a congruence class.
\end{lem}
\begin{proof}
($\Leftarrow$) Assume $\Theta$ is such a congruence.  Fix 
  $k<\omega$,
  $t^{\bA}\in \Clo_{k+1}\bA$, 
  $u, v \in A$, and
  $\bx, \by \in A^k$.
  We will prove the implication~(\ref{eq:22}), which in the present context is
\begin{equation*}
t^\bA(\bx,u) = t^\bA(\by,u) \quad \Longrightarrow \quad 
t^{\bA}(\bx,v) = t^{\bA}(\by,v).
\end{equation*}
Since $D(A)$ is a class of $\Theta$, we have 
  $(u,u) \mathrel{\Theta} (v,v)$, and since $\Theta$ is a reflexive relation, we have
  $(x_i,y_i)  \mathrel{\Theta} (x_i,y_i)$ for all $i$.  Therefore,
\begin{equation}
  \label{eq:90}  
  t^{\bA\times \bA}((x_1,y_1), \dots, (x_k,y_k), (u,u))
  \mathrel{\Theta}
  t^{\bA\times \bA}((x_1,y_1), \dots, (x_k,y_k), (v,v)).
\end{equation}
  since $t^{\bA \times \bA}$ is a term operation of $\bA\times \bA$.
  Note that~(\ref{eq:90}) is equivalent to
  \begin{equation}
    \label{eq:13}
    (t^{\bA}(\bx, u), t^{\bA}(\by,u))
    \mathrel{\Theta}
    (t^{\bA}(\bx, v), t^{\bA}(\by, v)).
  \end{equation}
  If $t^{\bA}(\bx, u)= t^{\bA}(\by, u)$ then 
  the first pair in~(\ref{eq:13}) belongs to the $\Theta$-class
  $D(A)$, so the second pair must also belong this $\Theta$-class.
  That is, $t^{\bA}(\bx, v)= t^{\bA}(\by, v)$, as desired.

  \vskip2mm

  \noindent ($\Rightarrow$) Assume $\bA$ is abelian. We show
  $\Cg^{\bA^2}(D(A)^2)$ has $D(A)$ as a block.  Assume
  \begin{equation}
    \label{eq:16}
  ((x,x), (c,c')) \in \Cg^{\bA^2}(D(A)^2).
  \end{equation}
  It suffices to prove that $c=c'$.  Recall, Malcev's congruence generation
  theorem states that (\ref{eq:16}) holds iff
  %$(x,x) \theta (c,c') \in \Cg^{\bA^2}(D(A)^2)$ iff %% for $0\leq i \leq n$ and 
  %% $0\leq j \leq n-1$, there exist
  \begin{align*}
  \exists \,& (z_0,z_0'), (z_1,z_1'), \dots, (z_n,z_n') \in A^2\\
    \exists \,& ((x_0,x_0'), (y_0,y_0')), ((x_1,x_1'), (y_1,y_1')), \dots, 
    ((x_{n-1},x_{n-1}'), (y_{n-1},y_{n-1}')) \in D(A)^2\\
    \exists \, & f_0, f_1, \dots, f_{n-1}\in F^*_{\bA^2}
  \end{align*}
  %% \begin{align*}
  %% (z_i,z_i') &\in A^2\\
  %% ((x_j,x_j'), (y_j,y_j')) &\in D(A)^2\\
  %% f_j &\in F^*_{\bA^2}
  %% \end{align*}
  such that 
  \begin{align}
    \label{eq:70}
    \{(x, x),(z_1,z_1')\} &= \{f_0(x_0,x_0'), f_0(y_0,y_0')\}\\
\nonumber
     \{(z_1,z_1'),(z_2,z_2')\} &= \{f_1(x_1,x_1'), f_1(y_1,y_1')\}\\
\nonumber
     & \vdots\\
    \label{eq:80}
     \{(z_{n-1},z_{n-1}'),(c, c')\} &= \{f_{n-1}(x_{n-1},x_{n-1}'), f_{n-1}(y_{n-1},y_{n-1}')\}
 \end{align}
The notation $f_i\in F^*_{\bA^2}$ means 
\begin{align*}
f_i(x, x') &= g_i^{\bA^2}((a_1, a_1'), (a_2, a_2'), \dots, (a_k, a_k'), (x, x'))\\
&= (g_i^{\bA}(a_1, a_2, \dots, a_k, x), g_i^{\bA}(a_1', a_2', \dots, a_k', x')),
\end{align*}
for some $g_i^{\bA} \in \Clo_{k+1}\bA$ and some constants 
$\ba = (a_1, \dots, a_k)$ and $\ba' = (a_1', \dots, a_k')$ in $A^k$. 
Now, $((x_i,x_i'), (y_i,y_i'))\in D(A)^2$ implies 
$x_i=x_i'$, and $y_i=y_i'$, so in fact we have 
\[
     \{(z_i,z_i'),(z_{i+1},z_{i+1}')\} = \{f_i(x_i,x_i), f_i(y_i,y_i)\} \quad (0\leq i < n).
\]
Therefore, by Equation~(\ref{eq:70}) we have either 
\[
     (x,x)= (g_i^{\bA}(\ba, x_0), g_i^{\bA}(\ba', x_0)) \quad \text{ or } \quad 
     (x,x)= (g_i^{\bA}(\ba, y_0), g_i^{\bA}(\ba', y_0)).
\]
Thus, either $g_i^{\bA}(\ba, x_0) =  g_i^{\bA}(\ba', x_0)$ %\quad \text{ or } \quad 
or $g_i^{\bA}(\ba, y_0) =  g_i^{\bA}(\ba', y_0)$.
By the abelian assumption, if one of these equations holds, then so does the
other. This and and Equation (\ref{eq:70}) imply $z_1 = z_1'$.  Applying the same
argument inductively, we find that $z_i = z_i'$ for all $1\leq i < n$ and so, by
(\ref{eq:80}) and the abelian property, we have $c= c'$.
\end{proof}

\begin{lem}{\bf\ref{lem:bijection_abelian}.}
Suppose $\rho: A_1 \to A_2$ is a bijection and suppose the graph
$\{(x, \rho x) \mid x \in A_1\}$ is a block of some congruence
$\beta \in \Con (A_1 \times A_2)$.  Then both $\bA_1$ and $\bA_2$ are abelian.
\end{lem}
\begin{proof}
  Define the relation $\alpha\subseteq (A_1\times A_1)^2$ as follows: for
  $((a,a'), (b,b')) \in (A_1\times A_1)^2$,
  \[
  (a,a')\mathrel{\alpha} (b,b')
  \quad \iff \quad
  (a, \rho a') \mathrel{\beta} (b, \rho b')
  \]
  We prove that the diagonal $D(A_1)$ is a block of $\alpha$ by showing that
  $(a, a) \mathrel{\alpha} (b,b')$ implies $b = b'$.
  Indeed, if $(a, a) \mathrel{\alpha} (b,b')$, then
  $(a, \rho a) \mathrel{\beta} (b, \rho b')$, which means that
  $(b, \rho b')$ belongs to the block and
  $(a, \rho a)/\beta = \{(x, \rho x): x\in A_1\}$.  Therefore,
  $\rho b  = \rho b'$, so $b = b'$ since $\rho$ is injective.
  This proves that $\bA_1$ is abelian.

  To prove $\bA_2$ is abelian, we reverse the roles of $A_1$ and $A_2$ in the
  foregoing argument.  
  If $\{(x, \rho x) \mid x \in A_1\}$ is a block of $\beta$,
  then 
  $\{(\rho^{-1}(\rho x), \rho x) \mid \rho x \in A_2\}$ is a block of $\beta$; that
  is, $\{(\rho^{-1} y, y) \mid y \in A_2\}$ is a block of $\beta$.  Define 
  the relation $\alpha\subseteq (A_2\times A_2)^2$ as follows: for
  $((a,a'), (b,b')) \in (A_2\times A_2)^2$,
  \[
  (a,a')\mathrel{\alpha} (b,b')
  \quad \iff \quad
  (\rho^{-1}a, \rho a') \mathrel{\beta} (\rho^{-1}b, \rho b').
  \]
  As above, we can prove that the diagonal $D(A_2)$ is a block of $\alpha$
  by using the injectivity of $\rho^{-1}$ to show that $(a, a) \mathrel{\alpha}
  (b,b')$
  implies $b = b'$.
\end{proof}

\subsection{Residuation Lemma}
\label{sec:residuation-lemma-1}
\begin{lem}{\bf\ref{lem:residuation}.} (Lem.~2.1 of \cite{overalgebras})\
\begin{enumerate}[\rm(i)]
  \item \label{item:residlemma-i} $^*\colon \Con\bB \rightarrow \Con\bA$ is a residuated mapping with
    residual $\resB$.
  \item \label{item:residlemma-ii} $\resB \colon  \Con\bA \rightarrow \Con\bB$ is a residuated mapping with
    residual $\hatmap$.
\item \label{item:residlemma-iii} For all $\alpha \in \Con\bA$, for all $\beta \in \Con\bB$,
\[
\beta = \alpha\resB \quad \Longleftrightarrow  \quad
\beta^* \leq \alpha \leq \widehat{\beta}.
\]
In particular,
$\beta^*\resB = \beta = \widehat{\beta}\resB$.
  \end{enumerate}
\end{lem}
\begin{proof}
  We first recall the definition of {\it residuated mapping}.  If $X$ and $Y$
  are partially ordered sets, and if
$f\colon  X \rightarrow Y$ and
$g\colon  Y \rightarrow X$ are order preserving maps, then the following are
equivalent:
\begin{enumerate}[(a)]
\item $f\colon  X \rightarrow Y$ is a {\it residuated mapping} with {\it residual}
$g\colon  Y \rightarrow X$;
\item 
$(\forall  x\in X)(\forall y\in Y) \, f(x) \leq y \iff x \leq g(y)$;
\item $g\circ f \geq \id_X$ and $f\circ g \leq \id_Y$,
\end{enumerate}
where $\id_S$ denotes the identity map on the set $S$.
The definition says that for each $y\in Y$ there is a unique
$x\in X$ that is maximal with respect to the property $f(x) \leq y$, and the
maximum is given by $x = g(y)$.
Thus, (\ref{item:residlemma-i}) is equivalent to:
$\forall  \alpha \in \Con\bA,\, \forall  \beta \in \Con\bB$,
\begin{equation}
  \label{eq:residuation}
\beta^* \leq \alpha \iff \beta \leq \alpha\resB.
\end{equation}
This is easily verified, as follows:  If
$\beta^* \leq \alpha$ and $(x,y)\in \beta$, then
$(x,y) \in \beta^* \leq \alpha$
and $(x,y) \in B^2$, so $(x,y)\in
\alpha\resB$.  If $\beta \leq \alpha\resB$ then
$\beta^* \leq (\alpha\resB)^* \leq \Cg^\bA(\alpha) = \alpha$.

Statement (\ref{item:residlemma-ii}) is equivalent to:
$\forall \alpha \in \Con\bA, \, \forall \beta \in \Con\bB$,
\begin{equation}
  \label{eq:resid2}
\alpha\resB\leq \beta \iff \alpha \leq \widehat{\beta}.
\end{equation}
This is also easy to check.  For, suppose
$\alpha\resB\leq \beta$ and $(x,y)\in \alpha$. Then $(ef(x), ef(y)) \in \alpha$
for all $f \in \Pol_1\bA$ and $(ef(x), ef(y)) \in B^2$, therefore,
$(ef(x), ef(y)) \in \alpha\resB \leq \beta$, so $(x,y) \in \widehat{\beta}$.
Suppose $\alpha \leq \widehat{\beta}$ and $(x,y) \in \alpha\resB$.
Then $(x,y) \in \alpha \leq  \widehat{\beta}$, so
$(ef(x), ef(y)) \in \beta$ for all $f\in \Pol_1\bA$, including $f=\id_A$, so
$(e(x), e(y)) \in \beta$. But $(x, y) \in B^2$, so $(x, y) = (e(x), e(y)) \in
\beta$.

Combining~(\ref{eq:residuation}) and~(\ref{eq:resid2}), we obtain statement (\ref{item:residlemma-iii}) of the lemma.
\end{proof}


\section{Example}
Let $X$ be a set.  It is useful to represent partitions of $X$ as
lists of lists, and write them as (possibly nonrectangular) arrays, where each
row represents a single block.  We do this in the following example, which 
aids our intuition when thinking about the Palfy-Saxl problem.

Let $X = \{0,1,2, \dots, 15\}$, and consider the equivalence relations
$\alpha_1, \dots, \alpha_5$ and $\beta$, generating the following sublattice of
$\Eq(X)$:

%% \begin{figure}
\begin{center}
  
      \begin{tikzpicture}[scale=1.2]
      \node (bot) at (0,0)  [draw, circle, inner sep=\dotsize] {};
      \node (top) at (0,3)  [draw, circle, inner sep=\dotsize] {};

      \node (a1) at (-2,1.5)  [draw, circle, inner sep=\dotsize] {};
      \node (a2) at (-1,1.5)  [draw, circle, inner sep=\dotsize] {};
      \node (a3) at (0,1.5)  [draw, circle, inner sep=\dotsize] {};
      \node (b) at (0,1)  [draw, circle, inner sep=\dotsize] {};
      \node (a4) at (1,1.5)  [draw, circle, inner sep=\dotsize] {};
      \node (a5) at (2,1.5)  [draw, circle, inner sep=\dotsize] {};

      \draw (top) node [above] {$1_X$};
      \draw (a1) node [left] {$\alpha_1$};
      \draw (a2) node [left] {$\alpha_2$};
      \draw (a3) node [right] {$\alpha_3$};
      \draw (b) node [right] {$\beta$};
      \draw (a4) node [right] {$\alpha_4$};
      \draw (a5) node [right] {$\alpha_5$};
      \draw (bot) node [below] {$0_X$};
      \draw[semithick] 
      (bot) -- (a1) -- (top) -- (a2) --
      (bot) -- (b) -- (a3) -- (top) -- (a4) --
      (bot) -- (a5) -- (top);
      \end{tikzpicture}
\end{center}
%%   \caption{The graph defining the relation $\tau(\alpha_1, \alpha_2, \beta)$;
%%     that is, $(x,y) \in \tau(\alpha_1, \alpha_2, \beta)$ if and only if there
%%     exist $a, b \in X$ satisfying the relations in the diagram.}
%%   \label{fig:rho}
%% \end{figure}
where $\alpha_1, \dots, \alpha_5$, and $\beta$ correspond to the following partitions of $X$:
\[
\begin{matrix}
& \alpha_1 &&\\
  [0 & 1 & 2 & 3]\\
  [4 & 5 & 6 & 7]\\
  [8 & 9 & 10 & 11]\\
  [12 & 13 & 14 & 15]\\
\end{matrix}
\qquad
\begin{matrix}
& \alpha_2 &&\\
  [0 &   4 &   8 & 12]\\
  [1 &   5 &   9 & 13]\\
  [2 &   6 & 10 & 14]\\
  [3 &   7 & 11 & 15]
\end{matrix}
\qquad
\begin{matrix}
& \alpha_3 &&\\
  [0 &   5 &  10 & 15]\\
  [1 &   4 &  11 & 14]\\
  [2 &   7 & 8 & 13]\\
  [3 &   6 & 9 & 12]
\end{matrix}
\]

\vskip5mm

\[
\begin{matrix}
& \alpha_4 &&\\
  [0 &   7 & 9 & 14]\\
  [1 &   6 & 8 & 15]\\
  [2 &   5 & 11 & 12]\\
  [3 &   4 & 10 & 13]\\
&&&\\
&&&\\
&&&
\end{matrix}
\qquad
\begin{matrix}
& \alpha_5 &&\\
  [0 &   6 & 11 & 13]\\
  [1 &   7 & 10 & 12]\\
  [2 &   4 & 9 & 15]\\
  [3 &   5 & 8 & 14]\\
&&&\\
&&&\\
&&&
\end{matrix}
\qquad
%% \begin{matrix}
%% & \beta &&\\
%%   [0 &   5 &  10 & 15]\\
%%  [1 &   4] & [11 & 14] \\
%%  [2 & 8] & [7 & 13]\\
%%  [3 &   12] & [6 & 9]
%% \end{matrix}
\begin{matrix}
& \beta &&\\
  [0 &   5 &  10 & 15]\\
&  [1 &   4] & \\
&  [2 & 8] &\\
&  [3 &   12] & \\
& [6 & 9] & \\
&  [7 & 13] &\\
& [11 & 14] & 
\end{matrix}
\]

The relations $\alpha_1, \dots, \alpha_5$ are \pppc.  Also, for 
each $\alpha_i$, with $i\neq 3$, it's clear that $\beta$ and $\alpha_i$ are
nonpermuting complements.
Here are some other facts that aid intuition.
\begin{fact}
  Each $M_3$ sublattice with all $\alpha$'s for atoms is a congruence lattice.  In
  other words, if $i$, $j$, $k$ are three distinct numbers in $\{1,2,\dots, 5\}$, then the sublattice $\{0_X, \alpha_i, \alpha_j,
  \alpha_k, 1_X\}$ is closed. 
\end{fact}
\begin{fact}
Consider any $M_4$ having all $\alpha$'s for atoms.  The closure is the $M_5$
lattice $\{0_X, \alpha_1, \dots, \alpha_5, 1_X\}$.
\end{fact}
\begin{fact}
  Each $M_4$ generated by $\beta$ and three $\alpha$'s complementary to
  $\beta$ is not closed.  The closure will have many relations in it.
\end{fact}
Regarding the last fact, I've forgotten how many relations are in the closure.

TODO: Check this; also check whether $\alpha_3$ and the other omitted $\alpha$
  always end up in the closure.
\begin{fact}
  The $M_3$ sublattice $\{0_X, \alpha_1, \alpha_2, \beta, 1_X\}$ is closed.
\end{fact}
  \begin{fact}
\label{fact:tau}
  The relation $\tau = \tau(\alpha_1, \alpha_2, \beta)$ defined via the
  Wheatstone Bridge (Figure~\ref{fig:rhoagain}) is a subset of $\beta$.
  \end{fact}

What follows is an informal discussion of the motivation that led to the
relation $\beta$ given in this example.  (This and other parts of
the Appendix are verbose and inelegant; all of this will be removed
eventually.) 

Regarding Fact~\ref{fact:tau}, $\beta$ was constructed specifically to provide a
nontrivial example where this fact might hold.  That is, we wanted to know if
an example existed in which $\beta$ has smaller height than $\alpha_i$ (so that 
$|x/\beta| \leq |y/\alpha_i| < |X/\beta|$, and so $\beta$ would not permute
with $\alpha_1$ and $\alpha_2$), and such that 
$\tau(\alpha_1, \alpha_2, \beta)\subseteq \beta$, so that the Wheatstone
Bridge of Figure~\ref{fig:rhoagain} would not generate an equivalence
relation that isn't already contained in 
$\{0_X, \alpha_1, \alpha_2, \beta, 1_X\}$.

To construct $\beta$, we started by assuming $0/\beta = \{0,5,10,15\}$,
which is the main diagonal of both $\alpha_1$ and $\alpha_2$.  Then we considered the
Wheatstone Bridge involving $\alpha_1$ and $\alpha_2$ and noticed that, if
$\tau \subseteq \beta$, then $\beta$ must contain all pairs that are at
``opposite corners'' (defined below)
relative to pairs on the main diagonal
$\{0,5,10,15\}$. 

%% Such a $\beta$ 
%% acheive this even when the remaining
%% $\beta$-blocks, other than $0/\beta$, have size 2, we have found an example that
%% satisfies the goal: does not permute with $\alpha_1$ or $\alpha_2$ and has
%% more blocks.

By ``opposite corners'' we mean the following.  Fix a pair in $\beta$, say,
$(0,10)\in \beta$, and consider the squares this pair generates in $\alpha_1$
and $\alpha_2$; that is, the squares with 0 and 10 at diagonal corners. We see
that 2 and 8 appear at the remaining corners of such squares.  We call the
corners labeled 2 and 8 the ``opposite corners'' relative to 0 and 10. 

The relation $\tau$ defined by the Wheatstone Bridge satisfies 
\[
0 \rbeta 10 \quad \longrightarrow \quad 2\rtau 8,
\]
and, by symmetry of $\alpha_1$ and $\alpha_2$,
\[
2 \rbeta 8 \quad \longrightarrow \quad 0\rtau 10.
\]

Let us make this more general and precise.
Recall the relation $\tau=\tau(\alpha_1, \alpha_2, \beta)\subseteq X\times X$ is
defined by
\begin{equation}
\label{eq:5}  
x \mathrel{\tau} y \quad \longleftrightarrow \quad (\exists (a, b) \in \beta)\;  
x \ralpha_1 a \ralpha_2 y \ralpha_1 b \ralpha_2 x.
\end{equation}
Graphically, $x \mathrel{\tau} y$ if and only if there exist $a, b \in X$
satisfying the relations depicted in Figure~\ref{fig:rhoagain}.  

Let us order the elements of the equivalence classes of $\alpha_1$
and $\alpha_2$ according to the row-column arrangements given in the array
representations above, and
denote by $\alpha_1(i,j)$ the $j$-th element of the $i$-th equivalence class of
$\alpha_1$---that is $\alpha_1(i,j)$ is the element in row $i$ and column $j$ of
the array representation of $\alpha_1$.
%% We know that operations respecting $\alpha_1$, $\alpha_2$, and $\beta$ must also
%% respect $\tau$.  


\begin{figure}
      \begin{tikzpicture}[scale=1.2]
      \node (x) at (-1,1)  [draw, circle, inner sep=\dotsize] {};
      \node (y) at (1,1)  [draw, circle, inner sep=\dotsize] {};
      \node (a) at (0,2)  [draw, circle, inner sep=\dotsize] {};
      \node (b) at (0,0)  [draw, circle, inner sep=\dotsize] {};
      \draw (x) node [left] {$x$};
      \draw (a) node [above] {$a$};
      \draw (b) node [below] {$b$};
      \draw (y) node [right] {$y$};
      \draw[semithick] (b)-- (a) node[pos=.5,right] {$\beta$};
      \draw[semithick] (x)-- (a) node[pos=.5,left] {$\alpha_1$};
      \draw[semithick] (x)-- (b) node[pos=.5,left] {$\alpha_2$};
      \draw[semithick] (b)-- (y) node[pos=.5,right] {$\alpha_1$};
      \draw[semithick] (a)-- (y) node[pos=.5,right] {$\alpha_2$};
      \end{tikzpicture}
  \caption{The Wheatstone Bridge which defines the relation 
    $\tau(\alpha_1, \alpha_2, \beta)$ as follows: 
    $(x,y) \in \tau(\alpha_1, \alpha_2, \beta)$ if and only if there 
    exist $a, b \in X$ satisfying the relations in the diagram.}
  \label{fig:rhoagain}
\end{figure}

Consider the Wheatstone Bridge diagram and note that, if 
$(x,y)$ and $(a, b)$ satisfy this diagram, so that~(\ref{eq:5}) holds, then  we have
\begin{equation}
\label{eq:6}  
x \in a/\alpha_1 \cap b/\alpha_2 \quad \text{ and } \quad 
y \in b/\alpha_1 \cap a/\alpha_2.
\end{equation}
Suppose $a = \alpha_1(i,j)$ and $b = \alpha_2(k,\ell)$.  Then, by~(\ref{eq:6}),  $x$ is the
point where the $i$-th row of $\alpha_1$ intersects the $k$-th row of
$\alpha_2$.  But notice that, in this example, the array representing $\alpha_2$
happens to be the transpose of the array representing $\alpha_1$.  Therefore,
the $k$-th row of $\alpha_2$ is the $k$-th column of $\alpha_1$, so $x$ is
the element contained in the $i$-th row and $k$-th column of $\alpha_1$, that is, 
$x = \alpha_1(i,k)$. Similarly, $y = \alpha_1(j,\ell)$.  More generally, for all
$i$, $j$, $r$, $s$ in $\{1, 2, 3, 4\}$, we have
\[
\alpha_1(i,j) \rbeta \alpha_1(r,s) \quad \longrightarrow \quad 
\alpha_1(i,s) \rtau \alpha_1(j,r).
\]
For example, looking at the array representing $\alpha_1$, we see that if, say,
$(2,15)$ were to belong to $\beta$, then the pair $(3,15)$ at 
the opposite corners must belong to $\tau(\alpha_1, \alpha_2, \beta)$.

%% \vskip2cm

%% \section{List of Acronyms}
%% \begin{acronym}
%% \acro{PPPC}{pairwise-permuting pairwise-complements}
%% \end{acronym}

%% \bibliographystyle{plainurl}
%% \bibliography{wjd}

\printbibliography


\end{document}


\[
\begin{matrix}
  \phantom{0}0 &   \phantom{0}1 &   \phantom{0}2 &   \phantom{0}3\\
    \phantom{0}4 &   \phantom{0}5 &   \phantom{0}6 &   \phantom{0}7\\
    \phantom{0}8 &   \phantom{0}9 & 10 & 11\\
  12 & 13 & 14 & 15
\end{matrix}
\qquad
\begin{matrix}
  \phantom{0}0 &   \phantom{0}4 &   \phantom{0}8 & 12\\
  \phantom{0}1 &   \phantom{0}5 &   \phantom{0}9 & 13\\
  \phantom{0}2 &   \phantom{0}6 & 10 & 14\\
  \phantom{0}3 &   \phantom{0}7 & 11 & 15
\end{matrix}
\qquad
\begin{matrix}
  \phantom{0}0 &   \phantom{0}5 &  10 & 15\\
  \phantom{0}1 &   \phantom{0}4 &  11 & 14\\
  \phantom{0}2 &   \phantom{0}7 & \phantom{0}8 & 13\\
  \phantom{0}3 &   \phantom{0}6 & \phantom{0}9 & 12
\end{matrix}
\qquad
\begin{matrix}
  \phantom{0}0 &   \phantom{0}7 & \phantom{0}9 & 14\\
  \phantom{0}1 &   \phantom{0}6 & \phantom{0}8 & 15\\
  \phantom{0}2 &   \phantom{0}5 & 10 & 12\\
  \phantom{0}3 &   \phantom{0}4 & 11 & 13
\end{matrix}
\qquad
\begin{matrix}
  \phantom{0}0 &   \phantom{0}6 & 11 & 13\\
  \phantom{0}1 &   \phantom{0}7 & 10 & 12\\
  \phantom{0}2 &   \phantom{0}4 & \phantom{0}9 & 15\\
  \phantom{0}3 &   \phantom{0}5 & \phantom{0}8 & 14
\end{matrix}
\]
and $\beta$ is 
\[
\begin{matrix}
  \phantom{0}0 &   \phantom{0}5 &  10 & 15\\
&  \phantom{0}1 &   \phantom{0}4 & \\
&  \phantom{0}2 & \phantom{0}8 &\\
&  \phantom{0}3 &   12 & \\
& \phantom{0}6 & \phantom{0}9 & \\
&  \phantom{0}7 & 13 &\\
& 11 & 14 & 
\end{matrix}
\]
