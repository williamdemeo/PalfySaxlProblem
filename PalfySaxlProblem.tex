%% Based on a TeXnicCenter-Template by Gyorgy SZEIDL.
%%%%%%%%%%%%%%%%%%%%%%%%%%%%%%%%%%%%%%%%%%%%%%%%%%%%%%%%%%%%%

%------------------------------------------------------------
%
\documentclass{amsart}
%
%----------------------------------------------------------
% This is a sample document for the AMS LaTeX Article Class
% Class options
%        -- Point size:  8pt, 9pt, 10pt (default), 11pt, 12pt
%        -- Paper size:  letterpaper(default), a4paper
%        -- Orientation: portrait(default), landscape
%        -- Print size:  oneside, twoside(default)
%        -- Quality:     final(default), draft
%        -- Title page:  notitlepage, titlepage(default)
%        -- Start chapter on left:
%                        openright(default), openany
%        -- Columns:     onecolumn(default), twocolumn
%        -- Omit extra math features:
%                        nomath
%        -- AMSfonts:    noamsfonts
%        -- PSAMSFonts  (fewer AMSfonts sizes):
%                        psamsfonts
%        -- Equation numbering:
%                        leqno(default), reqno (equation numbers are on the right side)
%        -- Equation centering:
%                        centertags(default), tbtags
%        -- Displayed equations (centered is the default):
%                        fleqn (equations start at the same distance from the right side)
%        -- Electronic journal:
%                        e-only
%------------------------------------------------------------
% For instance the command
%          \documentclass[a4paper,12pt,reqno]{amsart}
% ensures that the paper size is a4, fonts are typeset at the size 12p
% and the equation numbers are on the right side
%
\usepackage{amsmath}%
\usepackage{amsfonts}%
\usepackage{amssymb}%
\usepackage{mathrsfs}
%(wjd) added stmaryrd and enumerate packages
\usepackage{stmaryrd,enumerate}
\usepackage{graphicx}
\usepackage{comment}
\usepackage{tikz}
\usepackage{xspace}
\usepackage[printonlyused]{acronym}%
%------------------------------------------------------------
% Theorem like environments
%
\theoremstyle{plain}
\newtheorem{theorem}{Theorem}
\theoremstyle{definition}
\newtheorem{acknowledgement}{Acknowledgement}
\newtheorem{algorithm}{Algorithm}
\newtheorem{axiom}{Axiom}
\newtheorem{case}{Case}
\newtheorem{claim}{Claim}
\newtheorem{conclusion}{Conclusion}
\newtheorem{condition}{Condition}
\newtheorem{conjecture}{Conjecture}
\newtheorem{corollary}{Corollary}
\newtheorem{criterion}{Criterion}
\newtheorem{definition}{Definition}
\newtheorem{example}{Example}
\newtheorem{exercise}{Exercise}
\newtheorem{lemma}{Lemma}
\newtheorem{fact}{Fact}[section]
\newtheorem{notation}{Notation}
\newtheorem{problem}{Problem}
\newtheorem{proposition}{Proposition}
\newtheorem{remark}{Remark}
\newtheorem{solution}{Solution}
\newtheorem{summary}{Summary}
\newtheorem{tool}{Tool}
\theoremstyle{definition}
\newtheorem{Example}{Example}
\numberwithin{equation}{section}
%--------------------------------------------------------
%(wjd) A few useful macros 
\renewcommand{\phi}{\ensuremath{\varphi}}
\newcommand{\core}{\ensuremath{\operatorname{core}}}
\newcommand{\Sub}{\ensuremath{\operatorname{Sub}}}
\newcommand{\<}{\ensuremath{\langle}}
\renewcommand{\>}{\ensuremath{\rangle}}
\newcommand{\lb}{\ensuremath{\llbracket}}
\newcommand{\rb}{\ensuremath{\rrbracket}}
%(wjd) I think it's better to use \lb H, G \rb than [H,G], 
%      since single brackets typically denotes the commutator.
%      If you don't like this, uncomment the next two lines and
%      everything will change back to single brackets:
% \renewcommand{\lb}{\ensuremath{[}}
% \renewcommand{\rb}{\ensuremath{]}}

%
%(wjd) I prefer angled less or equal symbols, but you can
%      change these back by commenting the next 6 lines.
\renewcommand{\leq}{\ensuremath{\leqslant}}
\renewcommand{\nleq}{\ensuremath{\nleqslant}}
\renewcommand{\geq}{\ensuremath{\geqslant}}
\renewcommand{\lneq}{\ensuremath{\lneqslant}}
\renewcommand{\gneq}{\ensuremath{\gneqslant}}
\renewcommand{\ngeq}{\ensuremath{\ngeqslant}}
%
%(wjd) I prefer angled "normal or equal" symbol, but you can
%      change this back by commenting the next line.
\renewcommand{\unlhd}{\ensuremath{\trianglelefteqslant}}


\newcommand{\Peter}{P{\'e}ter}
\newcommand{\Palfy}{P\'alfy}
\newcommand{\bA}{\ensuremath{\mathbf{A}}}
\newcommand{\sP}{\ensuremath{\mathscr{P}}}
%\newcommand{\ker}{\ensuremath{\operatorname{ker}}}
\newcommand{\Con}{\ensuremath{\operatorname{Con}}}
\newcommand{\Eq}{\ensuremath{\operatorname{Eq}}}
\newcommand{\rel}{\ensuremath{\mathrel}}
\newcommand{\meet}{\ensuremath{\wedge}}
\newcommand{\join}{\ensuremath{\vee}}
\newcommand{\Meet}{\ensuremath{\bigwedge}}
\renewcommand{\Join}{\ensuremath{\bigvee}}
\newcommand{\nb}[1]{\ensuremath{\#\mathrm{Blocks}(#1)}}
%\newcommand{\nb}[1]{\ensuremath{\#{#1}}}

%-------------------------------------------------------
\begin{document}
\title{On a problem of P\'{a}lfy and Saxl}
\author{William DeMeo}
\date{November 13, 2013}

\maketitle

\section{Introduction}
In the paper \cite{PalfySaxl}, \Peter\ \Palfy\ and Jan Saxl pose the following 
\begin{quote}
  {\sc Problem.} Let $\bA$ be a finite algebra with $\Con \bA \cong M_n$, $n\geq
  4$. If three nontrivial congruences of $\bA$ pairwise permute, does it follow
  that every pair of congruences of $\bA$ permute?
\end{quote}

These notes collect some notation and facts that might be useful for attacking this problem.
Throughout, $X$ denotes a finite set, $\Eq(X)$ denotes the lattice of
equivalence relations on $X$ and, for $\alpha \in \Eq(X)$ and $x\in X$, we
denote by $x/\alpha$ the equivalence class of $\alpha$ containing $x$.
We often refer to equivalence classes as
``blocks,'' and we denote by $\nb{\alpha}$ the number of blocks of the
equivalence relation $\alpha$. 

For a given $\alpha\in \Eq(X)$ the map 
 $\phi_\alpha: x \mapsto x/\alpha$ is a
function from $X$ into the power set $\sP(X)$
with kernel $\ker \phi_\alpha = \alpha$. 
The \emph{block-size function} $x \mapsto |x/\alpha|$ is a function from $X$ into $\{1,2,\dots, |X|\}$.

We will often abuse notation and equate an equivalence relation with the
corresponding partition of the set $X$.  For example, we will equate 
the relation 
\[
\alpha = \{(0,0), (1,1), (2,2), (3,3), (0,1), (1,0), (2,3), (3,2)\}
\]
with the partition $|0,1|2,3|$, and often we resort to writing $\alpha = |0,1|2,3|$.

We say that $\alpha$ has \emph{uniform blocks} if 
all blocks of  $\alpha$ have the same size; or, equivalently,
the block-size function is constant: for all 
$x, y \in X$, $|x/\alpha| = |y/\alpha|$.  We will use $|x/\alpha|$, without specifying
a particular $x\in X$, to denote this block size.
%% \footnote{Alternatively, we
%%   might consider using 
%%   $|x_{\cdot}/\alpha|$ to emphasize that every $x\in X$ can be substituted for
%%   $x_\cdot$ without changing the value of $|x_\cdot/\alpha|$, but this notation
%%   may be too cumbersome.}  
Thus, when $\alpha$
has uniform blocks, we have $|X|= |x/\alpha| \cdot \nb{\alpha}$.

%% If $\alpha$ has uniform blocks of size $r$, then the number of blocks 
%% of $\alpha$ is $m = |X|/r$.  
We say that two equivalence relations with uniform
blocks  have 
%% If $\beta$ is another equivalence relation, we say
%% that $\alpha$ and $\beta$ have 
\emph{complementary uniform block structure}, or simply \emph{complementary
  blocks}, if the number of blocks of one is equal to
the block size of the other. In other words, if $\alpha$ and $\beta$ are two
equivalence relations on $X$ with uniform block sizes $|x/\alpha|$ and
$|x/\beta|$, respectively, then $\alpha$ and $\beta$ have complementary blocks
if and only if $(\forall x)(\forall y)\, |x/\alpha|\cdot |y/\beta| = |X|$.

%% If $\alpha \in \Eq(X)$ has uniform blocks of size $r$, we say that $\alpha$ has
%% \emph{balanced uniform block structure}, or simply \emph{balanced blocks},
%% provided $r = |X|^{1/2}$.

Given two equivalence relations $\alpha$ and $\beta$ on $X$, the relation
\[
\alpha \circ \beta = \{(x,y) \in X^2: (\exists z)x\rel{\alpha} z
\rel{\beta} y\}
\]
is called the \emph{composition of $\alpha$ and $\beta$}, and if 
$\alpha \circ \beta = \beta \circ \alpha$ then $\alpha$ and $\beta$ are said to
\emph{permute}, or to be \emph{permuting} equivalence relations.  
Note that $\alpha \circ \beta \subseteq \alpha \join \beta$ with equality if and
only if $\alpha$ and $\beta$ permute.

The largest and smallest equivalence relations on $X$ are $1_X = X^2$
and $0_X = \{(x,x) : x \in X\}$, respectively.

One more piece of shorthand notation will be useful below.  
Suppose $\Theta$ is a set of equivalence relations that are 
\emph{pairwise permuting pairwise complements} (\ac{PPPC})---that is, 
for all $\gamma \neq \delta$ in $\Theta$, we have
\[
\gamma \circ \delta = \delta \circ \gamma,  
\quad  \gamma \meet \delta = 0_X, 
\quad  \gamma \join \delta = 1_X.
\]
Then we say that the relations in $\Theta$ are \acs{PPPC}.

\section{Basic observations}
%% denote the relation $(\forall x)(\forall y)(x,y) \in 0_X \leftrightarrow x=y$.
We say that $\alpha$ and $\beta$ are \emph{complementary} equivalence relations
on $X$ provided $\alpha \join \beta = 1_X$ and $\alpha \meet \beta = 0_X$.
\begin{lemma}
\label{lem:1}
Suppose $\alpha$ and $\beta$ are complementary equivalence relations on
$X$. Then $\alpha$ and $\beta$ permute if and only if they have complementary
blocks.  That is,
\[
\alpha \circ \beta =1_X \quad \Longleftrightarrow \quad (\forall x)(\forall y)\,
|x/\alpha|\cdot |y/\beta| = |X|.
\]
\end{lemma}
\begin{corollary}
\label{cor:1}
Suppose $\alpha_1$, $\alpha_2$, $\alpha_3$ are pairwise complementary
equivalence relations on $X$. Then  $\alpha_1$, $\alpha_2$, $\alpha_3$
pairwise permute if and only if they all have uniform blocks of size 
$\sqrt{|X|}$.  In other words,
\[
(\forall i)(\forall j) \, (i\neq j \longrightarrow \alpha_i \circ \alpha_j = 1_X)
\quad \Longleftrightarrow \quad (\forall i)(\forall x) \, |x/\alpha_i| =
\sqrt{|X|}.
\]
In this case, we clearly have $|x/\alpha_i| =\nb{\alpha_i}$. 
\end{corollary}
\begin{proof}[Proof of Lemma~\ref{lem:1}]
  Assume $\alpha \circ \beta = \alpha \join \beta = 1_X$. Then, for all $x\in X$
  we have
  \begin{equation}
    \label{eq:1}
%x/(\alpha\circ \beta) = \dot{\bigcup}_{y \in x/\alpha} y/\beta = X,
x/(\alpha\circ \beta) = \coprod_{y \in x/\alpha} y/\beta = X,
  \end{equation}
where $\coprod$ denotes disjoint union.  The union is disjoint since
$\alpha \meet \beta = 0_X$.  Since the union in~(\ref{eq:1}) is all of $X$,
every block of $\beta$ must appear in the union, so the block $x/\alpha$ has
exactly $\nb{\beta}$ elements. Since $x$ was arbitrary, $\alpha$ has uniform
blocks of size $|x/\alpha| =\nb{\beta}$. Similarly, 
$x/(\beta \circ \alpha) = \coprod_{y \in x/\beta} y/\alpha = X$, so 
$|x/\beta| =\nb{\alpha}$ holds for all $x\in X$.  Therefore, 
for all $x, y\in X$, we have
\[
|x/\alpha|\cdot |y/\beta| 
 = |x/\alpha|\cdot \nb{\alpha} = |X|.
\]

To prove the converse, suppose $\alpha$ and $\beta$ are pairwise complements
with complementary blocks.  Then $|x/\alpha|\cdot |y/\beta| = |X|$, 
thus $|y/\beta| = |x/\alpha|^{-1} \cdot |X|  = \nb{\alpha}$ 
hold for all $x, y \in X$.  Therefore, for all $x\in X$, we have
\begin{align*}
\bigl|x/(\alpha\circ \beta)\bigr| &= \bigl|\coprod_{y\in x/\alpha} y/\beta\bigr|
= \sum_{y \in x/\alpha} |y/\beta|\\
&= \sum_{y \in x/\alpha} \nb{\alpha} \\
&= |x/\alpha|\nb{\alpha} = |X|.
\end{align*}
This proves that $\alpha\circ \beta = 1_X$, as desired.
\end{proof}

\begin{proof}[Proof of Corollary~\ref{cor:1}]
Since $\alpha_1$ and $\alpha_2$ permute and are complements, Lemma~\ref{lem:1}
implies they have complementary blocks, so
\begin{equation}
  \label{eq:2}
|x/\alpha_1| = |x/\alpha_2|^{-1} \cdot |X|= \nb{\alpha_2}.
\end{equation}
(This holds for all $x\in X$. Recall that complementary blocks are always
uniform.) 
Similarly, since $\alpha_1$ and $\alpha_3$ permute, we have
$|x/\alpha_1| = |x/\alpha_3|^{-1} \cdot |X| = \nb{\alpha_3}$.
Therefore, $\nb{\alpha_2} = \nb{\alpha_3}$.  Since 
$\alpha_2$ and $\alpha_3$ permute, we have
\begin{equation}
  \label{eq:3}
|x/\alpha_2| = |x/\alpha_3|^{-1} \cdot |X|= \nb{\alpha_3},
\end{equation}
and the latter is equal to $\nb{\alpha_2}$.  Therefore,
\[
|X| = |x/\alpha_2| \cdot \nb{\alpha_2} = |x/\alpha_2| \cdot |x/\alpha_2|. 
\]
Thus, $|x/\alpha_2|  = \sqrt{|X|}$, so by (\ref{eq:2}) and (\ref{eq:3}) we have 
$|x/\alpha_i| = \sqrt{|X|} = \nb{\alpha_i}$ for $i = 1, 2, 3$.

The converse is obvious, since if $\alpha_i$ and $\alpha_j$ are
complementary equivalence relations on $X$ with 
$|x/\alpha_i| = \sqrt{|X|}$, then 
$\nb{\alpha_i} = \sqrt{|X|}$, so
$\alpha_i \circ \alpha_j = 1_X$.
\end{proof}

From Corollary~\ref{cor:1} we see that the \Palfy-Saxl problem can be stated as
\begin{quote}
  {\sc Problem.} Let $\bA$ be a finite algebra with $\Con \bA \cong M_n$, $n\geq
  4$. %If three nontrivial congruences of $\bA$ pairwise permute, does it follow
If three atoms of $\bA$ have Property~(\ref{eq:4}) below, does it follow
  that every atom has Property~(\ref{eq:4})?
  \begin{equation}
    \label{eq:4}
(\forall x) \; |x/\alpha|  = \sqrt{|X|} = \nb{\alpha}
  \end{equation}
\end{quote}

\bigskip

%% \noindent To prove that the answer is ``no,'' it would suffice to find a 
%% congruence lattice $\Con \bA \cong M_n$, $n\geq 4$, where 3 atoms have
%% Property~(\ref{eq:4})   
%% % $|x/\alpha| = \sqrt{|X|}$
%%  and one atom $\beta$ has $|x/\beta| < \sqrt{|X|} < \nb{\beta}$.
%% \bigskip

\noindent To prove that the answer is ``yes,'' it will suffice to prove that 
if $M_n\leq \Eq(X)$ has 3 atoms 
with Property~(\ref{eq:4}) and
an atom $\beta$ with $|x/\beta| < \sqrt{|X|}$, then this $M_n$ is not a
congruence lattice.  
%% For example, we might find a method which shows in
%% such cases that the closure of the lattice will always contain some 
%% $\beta < \theta < 1_X$.

\section{Graphical Compositions}
Suppose $\alpha_1$, $\alpha_2$, and $\alpha_3$ are pairwise permuting
pairwise complements (\ac{PPPC}) in $\Eq(X)$, and let $\beta\in \Eq(X)$ be
complementary to 
each 
$\alpha_i$, so that 
\[
L = \{0_X, \alpha_1, \alpha_2, \alpha_3, \beta, 1_X\} \cong M_4.
\]  
Define the relation $\tau=\tau(\alpha_1, \alpha_2, \beta)\subseteq X\times X$ as
follows:
\[
x \mathrel{\tau} y \quad \longleftrightarrow \quad (\exists (a, b) \in \beta)\;  
x \rel{\alpha_1} a \rel{\alpha_2} y \rel{\alpha_1} b \rel{\alpha_2} x.
\]
Graphically, $x \mathrel{\tau} y$ if and only if there exist $a, b \in X$
satisfying the relations depicted in Figure~\ref{fig:rho}.

\newcommand\dotsize{1pt}
%\begin{center}
\begin{figure}
      \begin{tikzpicture}[scale=1.2]
      \node (x) at (-1,1)  [draw, circle, inner sep=\dotsize] {};
      \node (y) at (1,1)  [draw, circle, inner sep=\dotsize] {};
      \node (a) at (0,2)  [draw, circle, inner sep=\dotsize] {};
      \node (b) at (0,0)  [draw, circle, inner sep=\dotsize] {};
      \draw (x) node [left] {$x$};
      \draw (a) node [above] {$a$};
      \draw (b) node [below] {$b$};
      \draw (y) node [right] {$y$};
      \draw[semithick] (b)-- (a) node[pos=.5,right] {$\beta$};
      \draw[semithick] (x)-- (a) node[pos=.5,left] {$\alpha_1$};
      \draw[semithick] (x)-- (b) node[pos=.5,left] {$\alpha_2$};
      \draw[semithick] (b)-- (y) node[pos=.5,right] {$\alpha_1$};
      \draw[semithick] (a)-- (y) node[pos=.5,right] {$\alpha_2$};
      \end{tikzpicture}
  \caption{The graph defining the relation $\tau(\alpha_1, \alpha_2, \beta)$;
    that is, $(x,y) \in \tau(\alpha_1, \alpha_2, \beta)$ if and only if there
    exist $a, b \in X$ satisfying the relations in the diagram.}
  \label{fig:rho}
\end{figure}

%\end{center}
It is clear that $\tau$ is a a \emph{tolerance}, that is, a reflexive and symmetric binary relation.
Let $f\in X^X$ be a unary function and suppose that $f$ is \emph{compatible} with each
relation $\theta \in \{\alpha_1, \alpha_2, \beta\}$, that is, 
$(u,v)\in \theta \, \longrightarrow \, (f(u), f(v))\in \theta$.  Then $f$ is also
compatible with $\tau$. (Consider the diagram in Figure~\ref{fig:rho}, and give
each vertex $u$ the label $f(u)$.)

\begin{fact} If 
$L = \{0_X, \alpha_1, \alpha_2, \alpha_3, \beta, 1_X\} \cong  M_4$,
then
  \[
\alpha_1 \cap \tau(\alpha_1, \alpha_2, \beta)
= 0_X =  \alpha_2 \cap \tau(\alpha_1, \alpha_2, \beta),
\]
\[
\alpha_1 \cap \tau(\alpha_1, \alpha_3, \beta)
= 0_X =  \alpha_3 \cap \tau(\alpha_1, \alpha_3, \beta),
\]
\[
\alpha_2 \cap \tau(\alpha_2, \alpha_3, \beta)
 = 0_X = \alpha_3 \cap \tau(\alpha_2, \alpha_3, \beta).
\]
%%   \[
%% \tau(\alpha_1, \alpha_2, \beta) \cap \alpha_1
%% = 0_X = \tau(\alpha_1, \alpha_2, \beta) \cap \alpha_2,
%% \]
%% \[
%% \tau(\alpha_1, \alpha_3, \beta) \cap \alpha_1
%% = 0_X = \tau(\alpha_1, \alpha_3, \beta) \cap \alpha_3,
%% \]
%% \[
%% \tau(\alpha_2, \alpha_3, \beta) \cap \alpha_2
%%  = 0_X = \tau(\alpha_2, \alpha_3, \beta) \cap \alpha_3.
%% \]
\end{fact}
\begin{proof}
  Fix $(x,y) \in  \alpha_1 \cap \tau(\alpha_1, \alpha_2, \beta)$ and suppose 
  $a$ and $b$ satisfy the diagram in Figure~\ref{fig:rho}.  Then 
  $(x,y) \in \alpha_1$ implies $(a, b)\in \alpha_1 \meet \beta = 0_X$, so 
  $a = b$.  Therefore, $(x,y) \in \alpha_1 \meet \alpha_2 = 0_X$, so $x = y$.
  Proofs of the other identities are similar.
\end{proof}

\section{Functions Derived from Graphical Compositions}
Let $R_{12}^\beta$ be the relation on $X^2 \times X^2$ defined by 
\[
(a,b) \rel{R_{12}^\beta} (x,y) \quad \longleftrightarrow \quad 
(a,b) \in \beta \; \text{ and } \;
x \rel{\alpha_1} a \rel{\alpha_2} y \rel{\alpha_1} b \rel{\alpha_2} x.
\]
Define $R_{13}^\beta$ and $R_{23}^\beta$ similarly.  Graphically, 
$(a,b) \rel{R_{12}^\beta} (x,y)$ holds if and only if the relations in
Figure~\ref{fig:rho} are satisfied.
\begin{lemma}
\label{lem:injection}
Suppose $\alpha_i$ and $\alpha_j$ are complementary equivalence
relations on $X$ with uniform blocks of size $\sqrt{|X|}$.
Then the relation $R_{ij}^\beta$ restricted to $\beta\times X^2$ is
  a one-to-one function from $\beta$ into $X^2$.
%$R_{ij}^\beta : \beta   \rightarrow X^2$.
\end{lemma}
\begin{proof}
First we note that each pair $(a,b)\in \beta$ has at most one image. For if
$(a,b) \rel{R_{ij}^\beta} (x,y)$ and $(a,b) \rel{R_{ij}^\beta} (u,v)$, then 
$(x,u) \in  \alpha_i \meet \alpha_j = 0_X$ and 
$(y,v) \in  \alpha_i \meet \alpha_j = 0_X$, so $(x,y) = (u,v)$.

Next,
since both $\alpha_i$ and $\alpha_j$ have 
$\sqrt{|X|}$ blocks, and since each of these blocks has size $\sqrt{|X|}$, we
see that each block of $\alpha_i$ intersects each block of $\alpha_j$ at
exactly one point.  That is, for all $a, b \in X$, the set 
$a/\alpha_i \cap b/\alpha_j$ is a singleton.
Therefore, to each $(a,b)\in \beta$ there corresponds precisely one $(x,y)\in
X^2$ such that $(a,b) \rel{R_{ij}^\beta} (x,y)$ holds.  
Specifically, $\{x\} = a/\alpha_i \cap b/\alpha_j$ and 
$\{y\} = b/\alpha_i \cap a/\alpha_j$. 
Thus,  $R_{ij}^\beta$ is a function.  

From now on, we let 
$R_{ij}^\beta((a,b))$ denote the image of $(a,b)$ under $R_{ij}^\beta$; that
is, $R_{ij}^\beta((a,b))$ denotes the ordered pair $(x,y)$
satisfying $(a,b) \rel{R_{ij}^\beta} (x,y)$.


Suppose $R_{ij}^\beta((a,b)) = R_{ij}^\beta((c,d))$. Then 
$(a,c) \in \alpha_i\meet \alpha_j = 0_X$
and
$(b,d) \in \alpha_i\meet \alpha_j = 0_X$, so $(a,b) = (c,d)$.
Therefore,  $R_{ij}^\beta$ is one-to-one.
\end{proof}

If, in addition to the assumptions  of Lemma~\ref{lem:injection}, we assume that
the image of $\beta$ under $R_{ij}^\beta$ is contained in $\beta$, then 
$R_{ij}^\beta: \beta \rightarrow \beta$ is a bijective involution.
That is, $R_{ij}^\beta$ is one-to-one and onto, and 
$R_{ij}^\beta\circ R_{ij}^\beta$ is the identity map.
%% \begin{lemma}
%% Let $\{\alpha_i : 0\leq i < r\}$ be a set of pairwise complementary
%% equivalence relations on $X$.
%% \begin{enumerate}
%% \item If $\alpha_1 \circ \alpha_2 = \alpha_2 \circ \alpha_1$, then 
%% $\alpha_1$ and $\alpha_2$ have uniform blocks.
%% \item If $\alpha_1$ and $\alpha_2$ have uniform blocks of size $|X|^{1/2}$, then 
%% $\alpha_1 \circ \alpha_2 =  \alpha_2 \circ \alpha_1$.
%% \item Three pairwise complementary equivalence relations are pairwise permuting if and only
%%   if all three have uniform blocks of size $|X|^{1/2}$.
%% \end{enumerate}
%% these three
%% relations have complementary structure, and each
%% $\alpha_i$ has block size $|X|^{1/2} = n$, for some positive integer
%% $n$.  Thus, the number of blocks of each $\alpha_i$ is $|X|^{1/2}$, and
%% $|X| = n^2$.
%% \end{lemma}

%% To answer the Palfy-Saxl question affirmatively, it seems it would be enough to
%% show that if $L = \{0_X, \alpha_1, \dots, \alpha_{n-1}, \beta, 1_X\} \cong M_n$
%% is a congruence lattice and if $\alpha_1$, $\alpha_2$, and $\alpha_3$ are \PPPC, and if 
%%  $R_{ij}^\beta: \beta \rightarrow \beta$ for each $i\neq j$ in $\{1, 2, 3\}$,
%% then the congruence relation $\beta$ contains exactly $|\beta| = |X|^{3/2}$
%% ordered pairs, and thus has the same block structure as, and permutes with, $\alpha_i$ for $i\in
%% \{1,2,3\}$.

\section{Final piece of the puzzle}
As above, suppose $L = \{0_X, \alpha_1, \alpha_2, \alpha_3, \beta, 1_X\} \cong  M_4$ is a
congruence lattice with $\alpha_i$ \ac{PPPC}.  
Suppose $R_{ij}^\beta: \beta \rightarrow \beta$ holds for all 
$i, j \in \{1,2,3\}$.

\begin{lemma}
  \label{lem:missingpiece}
If %$(a,w) \in \alpha_1\circ \beta$ with 
$a \rel{\alpha_1} z \rel{\beta} w$,
then one of the following holds:
\begin{enumerate}
\item $(a,w) \in \alpha_2$, 
\item $(a,w) \in \alpha_3$, 
\item $(a,w) \in \beta$, 
\item $a/\alpha_2 \cap z/\alpha_3 \cap w/\alpha_1 \neq \emptyset$,
\item $a/\alpha_3 \cap z/\alpha_2 \cap w/\alpha_1 \neq \emptyset$.
\end{enumerate}
\end{lemma}

If Lemma~\ref{lem:missingpiece} is true, then we can prove the following:
\begin{theorem}
If $L = \{0_X, \alpha_1, \alpha_2, \alpha_3, \beta, 1_X\} \cong  M_4$ is a
congruence lattice with $\alpha_i$ \ac{PPPC}, then $\beta$ permutes with each $\alpha_i$.
\end{theorem}
\begin{proof}
We will show $\alpha_1 \circ \beta \subseteq \beta \circ \alpha_1$.
Assume $a \rel{\alpha_1} z \rel{\beta} w$.  We consider each of the cases in
Lemma~\ref{lem:missingpiece} in turn and,
in each case, find $b$ satisfying $a \rel{\beta} b \rel{\alpha_1} w$.
\begin{enumerate}
\item If $(a,w) \in \alpha_2$, then let $b = z/\alpha_2 \cap w/\alpha_1$.  Then
$R^\beta_{12}(z,w) = (a,b)$ and since 
$R^\beta_{12}: \beta \rightarrow \beta$, we have $(a,b) \in \beta$, so 
$a \rel{\beta} b \rel{\alpha_1} w$, as desired.  
\item If $(a,w) \in \alpha_3$, then let $b = z/\alpha_3 \cap w/\alpha_1$. Use the same
argument as in the first case, but replace $R^\beta_{12}$ with $R^\beta_{13}$. 
\item If $(a,w) \in \beta$, then let $b = a$. 
\item If  $a/\alpha_2 \cap z/\alpha_3 \cap w/\alpha_1 \neq \emptyset$, then let
  $y$ denote the element in this set.  Let $x = z/\alpha_1 \cap w/\alpha_3$, and
  let $b = x/\alpha_2\cap y/\alpha_1$.  
  Then $(R^\beta_{12}\circ R^\beta_{13})(z,w) = R^\beta_{12}(x,y) = (a,b)$, so $(a,b) \in \beta$.
  Now, $b\rel{\alpha_1} y \rel{\alpha_1} w$, so
  $a \rel{\beta} b \rel{\alpha_1} w$, as desired.
\item If $a/\alpha_3 \cap z/\alpha_2 \cap w/\alpha_1 \neq \emptyset$, then let 
  $y$ denote this element, let $x = z/\alpha_1 \cap w/\alpha_2$, and
  let $b = x/\alpha_3\cap y/\alpha_1$.  
  Then $(R^\beta_{13}\circ R^\beta_{12})(z,w) = R^\beta_{12}(x,y) = (a,b)$, so $(a,b) \in \beta$.
  Now, $b\rel{\alpha_1} y \rel{\alpha_1} w$, so $a \rel{\beta} b \rel{\alpha_1} w$, as desired.
\end{enumerate}
\end{proof}

\section{Proof of Lemma 3}
Consider the relation $\theta_{ij}$ defined as follows:
\[
x \rel{\theta_{ij}} y \quad \longleftrightarrow \quad (\exists a, b) \;
a \rel{\alpha_i} x \rel{\alpha_j} b \rel{\beta} y \rel{\alpha_j} a.
\]
Easy arguments similar to those above establish that 
\[
\theta_{ij} \cap \alpha_i = \theta_{ij} \cap \alpha_j =
\theta_{ij} \cap \beta = 0_X.\]
On the other hand, since $L$ is a congruence lattice, it must be the case that
the transtive closure of $\theta_{ij}$ is contained in $L$.\\
\\
{\it TODO: prove of Lemma 3 (if possible).}

\appendix
\section{Example}
Let $X$ be a set.  It is useful to represent partitions of $X$ as
lists of lists, and write them as (possibly nonrectangular) arrays, where each
row represents a single block.  We do this in the following example, which seems
to aid intuition when thinking about the Palfy-Saxl problem.

Let $X = \{0,1,2, \dots, 15\}$, and consider the equivalence relation
$\alpha_1, \dots, \alpha_5$ and $\beta$, generating the following sublattice of
$\Eq(X)$:

%% \begin{figure}
\begin{center}
  
      \begin{tikzpicture}[scale=1.2]
      \node (bot) at (0,0)  [draw, circle, inner sep=\dotsize] {};
      \node (top) at (0,3)  [draw, circle, inner sep=\dotsize] {};

      \node (a1) at (-2,1.5)  [draw, circle, inner sep=\dotsize] {};
      \node (a2) at (-1,1.5)  [draw, circle, inner sep=\dotsize] {};
      \node (a3) at (0,1.5)  [draw, circle, inner sep=\dotsize] {};
      \node (b) at (0,1)  [draw, circle, inner sep=\dotsize] {};
      \node (a4) at (1,1.5)  [draw, circle, inner sep=\dotsize] {};
      \node (a5) at (2,1.5)  [draw, circle, inner sep=\dotsize] {};

      \draw (top) node [above] {$1_X$};
      \draw (a1) node [left] {$\alpha_1$};
      \draw (a2) node [left] {$\alpha_2$};
      \draw (a3) node [right] {$\alpha_3$};
      \draw (b) node [right] {$\beta$};
      \draw (a4) node [right] {$\alpha_4$};
      \draw (a5) node [right] {$\alpha_5$};
      \draw (bot) node [below] {$0_X$};
      \draw[semithick] 
      (bot) -- (a1) -- (top) -- (a2) --
      (bot) -- (b) -- (a3) -- (top) -- (a4) --
      (bot) -- (a5) -- (top);
      \end{tikzpicture}
\end{center}
%%   \caption{The graph defining the relation $\tau(\alpha_1, \alpha_2, \beta)$;
%%     that is, $(x,y) \in \tau(\alpha_1, \alpha_2, \beta)$ if and only if there
%%     exist $a, b \in X$ satisfying the relations in the diagram.}
%%   \label{fig:rho}
%% \end{figure}
where $\alpha_1, \dots, \alpha_5$, and $\beta$ correspond to partitions of $X$
as follows:
\[
\begin{matrix}
& \alpha_1 &&\\
  [0 & 1 & 2 & 3]\\
  [4 & 5 & 6 & 7]\\
  [8 & 9 & 10 & 11]\\
  [12 & 13 & 14 & 15]\\
\end{matrix}
\qquad
\begin{matrix}
& \alpha_2 &&\\
  [0 &   4 &   8 & 12]\\
  [1 &   5 &   9 & 13]\\
  [2 &   6 & 10 & 14]\\
  [3 &   7 & 11 & 15]
\end{matrix}
\qquad
\begin{matrix}
& \alpha_3 &&\\
  [0 &   5 &  10 & 15]\\
  [1 &   4 &  11 & 14]\\
  [2 &   7 & 8 & 13]\\
  [3 &   6 & 9 & 12]
\end{matrix}
\]

\vskip5mm

\[
\begin{matrix}
& \alpha_4 &&\\
  [0 &   7 & 9 & 14]\\
  [1 &   6 & 8 & 15]\\
  [2 &   5 & 10 & 12]\\
  [3 &   4 & 11 & 13]\\
&&&\\
&&&\\
&&&\\
\end{matrix}
\qquad
\begin{matrix}
& \alpha_5 &&\\
  [0 &   6 & 11 & 13]\\
  [1 &   7 & 10 & 12]\\
  [2 &   4 & 9 & 15]\\
  [3 &   5 & 8 & 14]\\
&&&\\
&&&\\
&&&\\
\end{matrix}
\qquad
\begin{matrix}
& \beta &&\\
  [0 &   5 &  10 & 15]\\
&  [1 &   4] & \\
&  [2 & 8] &\\
&  [3 &   12] & \\
& [6 & 9] & \\
&  [7 & 13] &\\
& [11 & 14] & 
\end{matrix}
\]

The relations $\alpha_1, \dots, \alpha_5$ are pairwise
permuting pairwise complements (\acs{PPPC}).  Also, for 
each $\alpha_i$, with $i\neq 3$, it's clear that $\beta$ and $\alpha_i$ are
nonpermuting complements.
Here are some other facts which can may aid intuition.
\begin{fact}
  Each $M_3$ sublattice with all $\alpha$'s for atoms is a congruence lattice.  In
  other words, if $i$, $j$, $k$ are three distinct numbers from  the set 
  $\{1,2,\dots, 5\}^3$, then the sublattice $\{0_X, \alpha_i, \alpha_j,
  \alpha_k, 1_X\}$ is closed. 
\end{fact}
\begin{fact}
Consider any $M_4$ having all $\alpha$'s for atoms.  The closure is the $M_5$
lattice $\{0_X, \alpha_1, \dots, \alpha_5, 1_X\}$.
\end{fact}
\begin{fact}
  The $M_3$ sublattice $\{0_X, \alpha_1, \alpha_2, \beta, 1_X\}$ is closed.
\end{fact}
\begin{fact}
  Any $M_4$ generated by $\beta$ and three $\alpha$'s complementary to
  $\beta$ is not closed.  The closure will have many relations in it.
\end{fact}
Regarding the last fact, I've forgotten how many relations are in the closure.

TODO: Check this; also check whether $\alpha_3$ and the other omitted $\alpha$
  always end up in the closure.

\section{List of Acronyms}
\begin{acronym}
\acro{PPPC}{pairwise permuting pairwise complements}
\end{acronym}


\bibliographystyle{plainurl}
\bibliography{wjd}


\end{document}


\[
\begin{matrix}
  \phantom{0}0 &   \phantom{0}1 &   \phantom{0}2 &   \phantom{0}3\\
    \phantom{0}4 &   \phantom{0}5 &   \phantom{0}6 &   \phantom{0}7\\
    \phantom{0}8 &   \phantom{0}9 & 10 & 11\\
  12 & 13 & 14 & 15
\end{matrix}
\qquad
\begin{matrix}
  \phantom{0}0 &   \phantom{0}4 &   \phantom{0}8 & 12\\
  \phantom{0}1 &   \phantom{0}5 &   \phantom{0}9 & 13\\
  \phantom{0}2 &   \phantom{0}6 & 10 & 14\\
  \phantom{0}3 &   \phantom{0}7 & 11 & 15
\end{matrix}
\qquad
\begin{matrix}
  \phantom{0}0 &   \phantom{0}5 &  10 & 15\\
  \phantom{0}1 &   \phantom{0}4 &  11 & 14\\
  \phantom{0}2 &   \phantom{0}7 & \phantom{0}8 & 13\\
  \phantom{0}3 &   \phantom{0}6 & \phantom{0}9 & 12
\end{matrix}
\qquad
\begin{matrix}
  \phantom{0}0 &   \phantom{0}7 & \phantom{0}9 & 14\\
  \phantom{0}1 &   \phantom{0}6 & \phantom{0}8 & 15\\
  \phantom{0}2 &   \phantom{0}5 & 10 & 12\\
  \phantom{0}3 &   \phantom{0}4 & 11 & 13
\end{matrix}
\qquad
\begin{matrix}
  \phantom{0}0 &   \phantom{0}6 & 11 & 13\\
  \phantom{0}1 &   \phantom{0}7 & 10 & 12\\
  \phantom{0}2 &   \phantom{0}4 & \phantom{0}9 & 15\\
  \phantom{0}3 &   \phantom{0}5 & \phantom{0}8 & 14
\end{matrix}
\]
and $\beta$ is 
\[
\begin{matrix}
  \phantom{0}0 &   \phantom{0}5 &  10 & 15\\
&  \phantom{0}1 &   \phantom{0}4 & \\
&  \phantom{0}2 & \phantom{0}8 &\\
&  \phantom{0}3 &   12 & \\
& \phantom{0}6 & \phantom{0}9 & \\
&  \phantom{0}7 & 13 &\\
& 11 & 14 & 
\end{matrix}
\]
